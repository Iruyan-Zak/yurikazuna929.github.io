\documentclass[12pt]{jsarticle}
\usepackage[top=15truemm,bottom=15truemm,left=10truemm,right=10truemm]{geometry}
\usepackage[dvipdfmx]{graphicx}
\usepackage{amsmath}
\usepackage{txfonts}
\usepackage{mathptmx}
\usepackage{indentfirst}
\usepackage[svgnames]{xcolor}%tikzパッケージよりも前に読み込みます。
\usepackage{tikz}
\begin{document}
%\renewcommand{\[}{\begin{align}}
%\renewcommand{\]}{\end{align}}

\section{交流理論と電気回路}
直流 DC Direct Current

交流 AC Alternating current

正弦波交流
\begin{align}i=I_m\sin{(\omega t+\phi)}\end{align}

$i$:瞬時値
$I_m$:振幅
$\omega$:角周波数
$\phi$:初期位相

$f$:周波数、$T$:周期とすると、
\begin{align*}
\omega = 1\pi f,\;\;f=\frac1T
\end{align*}

\begin{align*}
I=\frac1{\sqrt2}Im,\;\;I:実効値\\
平均電力P&=\frac1T\int^T_0i^2Rdt\\
&=\frac{I_m^2}2R=I^2R
\end{align*}

\subsection{複素数表示}
\begin{align*}
I&=I_me^{j(\omega t+\phi)}\\
&=I_m\cos(\omega t+\phi)+jI_m\sin(\omega t+\phi) 
\end{align*}
元の正弦波は虚部に等しい。

\begin{align*}
\dot I=Ie^{j\phi}
\end{align*}

\[
\dot I=Ie^{j\phi}
\]
\end{document}
