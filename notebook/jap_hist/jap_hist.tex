\documentclass[12pt,fleqn]{ltjsarticle}
\usepackage{mymacro}
\usepackage{notemode}
\usepackage{mathmode}
\begin{document}
\clearpage
\section{自由民権運動}
\subsection{背景}
◎士族の不満
\begin{itemize}
\item 士族は自ら明治維新を推め、武士を解体していった
\item 反政府運動
\begin{itemize}
\item 武力:士族反乱
\item 言論:自由民権運動
\end{itemize}
\end{itemize}
明治六年の政変以降、政治家がこれらの運動に参加し、指導していった。


◎農民の不満
\begin{itemize}
\item 血税一揆
\item 地租改正反対一揆
\item 学制反対一揆
\end{itemize}

\subsection{自由民権運動}
\subsubsection{運動の開始}
◎政社の結成
\begin{itemize}
\item \bold{1874年 \red{愛国公党}を結成}:板垣退助・後藤象二郎ら

→\red{民撰議院設立建白書}
\item \bold{1874年 \red{立志社}}(高知):片岡健吉・板垣ら
\item \bold{1875年 \red{愛国社}}(大阪):立志社が全国組織を目指して設立
\end{itemize}
◎政府の対応
\begin{itemize}
\item \bold{1875年 \red{大阪会議}}:大久保・板垣・木戸など
\item 明治天皇が\red{漸次立憲政体樹立の詔}を発する
\item 立法機関として「元老院(上院)」、「地方官会(下院)」、
司法機関として「大審院」を設立
\item 言論統制を強化
\begin{itemize}
\item 1875年 讒謗律(政府の文句を紙面に乗せるとOUT)
\item 新聞紙条例
\end{itemize}
\end{itemize}

\subsubsection{運動の進展}
\bold{1877年 \red{立志社建白}}
\begin{itemize}
\item 国会開設
\item 地租軽減
\item 条約改正
\end{itemize}
これは政府に却下される。

この時、運動の支持者は士族から地主や商工業者にまで拡大していた。

\bold{1880年 \red{国会期成同盟}}(不受理)

これに対して政府は、集会を規制する\red{集会条例}を制定した。

\subsection{明治十四年の政変}
1878年。大久保が暗殺されてしまいました。

\begin{csvtab}{ccc}
候補, 国会開設の時期,立憲政治のモデル
大隈重信,早期(2年後),イギリス(政党政治)
伊藤博文,漸進的,プロイセン
\end{csvtab}
多数派であった伊藤博文が後継する。

\bold{1881年 \red{開拓使官有物払い下げ事件}}
\begin{itemize}
\item 開拓使(役所)を閉める際にお友達に官有物を
安く売り払った(えこひいきした)事件
\item 干ばつも相俟って民権派(世論)の批判が殺到
\item 大隈が民権派の肩を持つ
\item 政府の中で大隈が民権派を扇動したという噂が立つ
\item 伊藤ら多数派、天皇の力で大隈参議を罷免する
(\bold{\red{明治十四年の政変}})
\item とりあえず文句を言われないように払い下げは中止
\item \red{国会開設の勅諭[詔]}を出して1890年の国会開設を約束
\end{itemize}

\subsection{松方財政}
\subsubsection{背景}
\begin{align*}
&\begin{simul}
明治維新諸政策\\
西南戦争(1年分の予算を使用)
\end{simul}\rightarrow 政府は資金難になる\\
&\Rightarrow
\begin{simul}
国立銀行からの借り入れ(国立銀行券を発行して対応)\\
政府が自分たちで紙幣を発行
\end{simul}= 不換紙幣が大量に刷られた\\
&\begin{simul}
\Rightarrow {\rm \bold{インフレ}}\\
対して地租は定額
\end{simul}\rightarrow 政府はさらなる財政難に陥る。
\end{align*}

\subsubsection{松方財政}
1881年ごろ 大蔵卿 大隈重信の政策
\begin{itemize}
\item 増税:酒造税など、嗜好品に対して増税する
\item 官営事業(必須・かつ赤字)の払い下げに着手
\end{itemize}
→明治十四年の政変:大隈罷免

\bold{1881年 大蔵卿\red{松方正義}が政策を継承}
\begin{itemize}
\item 軍事費以外の歳出抑制
\item 増税:酒・たばこなど
\item 官営事業の払い下げを継続
\end{itemize}
$\rightarrow$余剰金(利益・黒字)が生まれる
\begin{align*}
\Rightarrow\begin{simul}
借入金返済(国立銀行側で償却)\\
政府紙幣償却
\end{simul}\rightarrow
インフレが落ち着くと同時にデフレに突入する
\end{align*}

◎政府紙幣償却について
\begin{itemize}
\item 政府は紙幣を発行しました = 神様からお金を借りました
\item 政府に黒字が生まれました = 返済する用意が出来ました
\item 政府は自分で発行した紙幣を処分しました = 神様にお金を返しました
\end{itemize}

\bold{1882年 \red{日本銀行}を設立}
\begin{itemize}
\item 国立銀行から国立銀行券を発行する機能を取り上げる
\item 代わりに銀兌換の日本銀行券を発券する機能を持っている
\end{itemize}

\subsubsection{影響}
デフレーション→不況
\begin{itemize}
\item インフレ時には景気が良くなる場合と
悪くなる場合(スタグフレーション)があり得る
\item しかしデフレ時に景気が良くなることはない
\begin{itemize}
\item 物価が継続的に下落する状況では、必要なものでない限りは
後で買うほうが得になる→消費低迷
\end{itemize}
\end{itemize}

これにより中小自作農の生活に打撃(お金は減るのに地租は変わらず)

\subsection{政党の結成}
\subsubsection{新政党}
\begin{itemize}
\item \bold{\red{自由党}(板垣) 1881\~}
\item \bold{\red{立憲改進党}(大隈) 1882\~}
\item \bold{\red{立憲帝政党}(福地源一郎) 1882\~}
\end{itemize}


\begin{csvtab}{cccc}
政党,支持層,モデル,スタンス,
自由党,旧士族・地主,フランス,国民主権・普通選挙など,
立憲改進党,都市部の実業家・知識人(福沢諭吉など),イギリス,君臣同治
立件帝政党,-,-,政府支持・国粋主義
\end{csvtab}

\subsubsection{私擬憲法}
民間の憲法草案

\subsection{自由民権運動の激化と挫折}
◎松方財政下の農村(不況)では
\begin{itemize}
\item 民権運動からの離脱者が現れる
\item 民権運動が超激化する

福島事件・秩父事件・大阪事件など
\end{itemize}

その結果政党指導者も望んでいない方向に運動が進み、
政党指導者が離脱することになる。

政党指導者がいない運動は政府が弾圧した。

\subsubsection{民権運動の再編}
\bold{\red{大同団結運動}(1887頃\~)}
星亨(ほし とおる)や後藤象二郎など、穏健派が指導

\bold{\red{三大事件建白運動}}
\begin{itemize}
\item 地租軽減
\item 言論集会の自由

新聞紙条例・讒謗律の撤廃
\item (不平等)条約の改正
\end{itemize}

\bold{1887年} これに対して政府は\bold{\red{保安条例}}を公布し、
民権運動家を追放する。

\clearpage
\section{立憲体制の確立}
◎列強にナメられない(一人前の)国にするならやっぱり議会は外せないよねー。

ここでいう「一人前の」とはまさしく「西洋風の」と同義で、
国際的地位を獲得するために、日本はどんどん西洋かぶれしていく。

\subsection{制度改革}
\subsubsection{準備}
\bold{1882年} 伊藤博文たちがヨーロッパに渡る。
\begin{itemize}
\item 君主権の強いドイツ・オーストリアの憲法理論などを学ぶ
\item もともと議会は王様の権力に対抗する組織のはず

\begin{itemize}
\item 日本は天皇を崇め奉っていたので、
天皇の権力はそのままに議会を作って国民を黙らせることができる
プロイセン憲法がちょうど良かった
\item イギリス式を否定しておきたい考えもあった
\end{itemize}
\end{itemize}

\bold{1883年} 帰国する。
\begin{itemize}
\item \bold{\red{制度取調局} (1884年)}
\begin{itemize}
\item 長官は伊藤博文
\item 伊藤博文はいろんな仕事を手広くやっていて、いろんなところに名前を残している
\end{itemize}
\end{itemize}

\bold{1884年 \red{華族令}}
\begin{itemize}
\item 国家に貢献した人たちを華族にしちゃおう
\item 国家に貢献した人=維新で活躍した人、
つまり政治家や軍人、のちには政商(実業家)も
\item ここで追加された華族は「新華族」と呼ばれた(若干の軽蔑を含む)
\item 華族は上院や貴族院の構成員となることを想定していたので、
これによって根回しをして、下院(衆議院)を牽制する狙いがあった
\end{itemize}

\bold{1885年 \red{内閣制度}}
\begin{itemize}
\item 首相と大臣から成る、簡素化した政府組織
\item 初代首相は伊藤博文(選挙で決まったわけではない)
\item 宮内省を政治と分離させる「宮中と府中の別」を発動
\begin{itemize}
\item 天皇は政治責任を負わず、代わりに行政からは手を引く
\end{itemize}
\item 官僚を育成する仕組みを作る
\begin{itemize}
\item 帝国大学令(1886年):帝国大学(今の東京大学)を作る
\item 文官高等試験を開始
\end{itemize}
\end{itemize}

\subsection{大日本帝国憲法}
\subsubsection{草案作成}
1886-87年 伊藤博文を中心に\red{枢密院}で審議し、秘密裏に起草
\begin{itemize}
\item 枢密院の初代議長も伊藤博文(このために首相も辞めている)
\end{itemize}

\subsubsection{発布}
\bold{1889年2月11日}、
(太古の日本に初めて統治者が現れた日と言い伝えられている紀元節の日を選んで)
\bold{\red{大日本帝国憲法}}を発布する。

○余談
主権者は自らの権力の拠り所を決めるのにいつも迷う。
日本の場合は、「万世一系」という考え方で、
天皇は神の血筋を引くものであると定義し、
神に権力の拠り所を求めている。
憲法の権力の拠り所はもちろん天皇で、
天皇の名のもとに紀元節を選んで、憲法を発布している。

\subsubsection{特色}
\begin{itemize}
\item 欽定憲法(天皇が天皇の名で発動した憲法)なのに天皇の権力を制限する内容である
\begin{itemize}
\item 書き出し:天皇は偉い
\item 途中:でも天皇の権力は一部制限するよ
\item 途中:国民の権利は守るよ
\end{itemize}

\item 条文による天皇の定義
\begin{itemize}
\item 第一条:最高の統治者
\item 第三条:「神聖にして不可侵」(政治責任は回避)
\item 第四条:憲法に基づいて統治する(立憲君主である)
\item 第五条:議会の「協賛」を以って立法権を持つ(議会の主権を肯定してる風)
\item 第十から十三条:天皇大権(議会が手を出せない特権)
\begin{itemize}
\item 第十条:文武官の任免(名前を貸すだけ)
\item 第十一条:陸海軍の統帥(後の戦争でひと悶着おこす)
\item 第十二条:軍の編成・予算の決定
\item 第十三条:宣戦布告・講和・条約(に名前を貸す)
\end{itemize}
\end{itemize}

\item 条文によるその他の定義
\begin{itemize}
\item 第五十五条:内閣の国務大臣は天皇が任命する(任命責任が天皇にある)
\begin{itemize}
\item 相談相手は元老(首相をしてた人たちとか)

元老の存在は憲法に記されていない
\end{itemize}
\item 第三十三条:議会は二院制をとる
\begin{itemize}
\item \red{衆議院}と\red{貴族院}とは対等である

\end{itemize}

\item 国民の権利は"法律の範囲で"保証
\item 国家機構
\begin{itemize}
\item 天皇の手の中に軍と枢密院がある
\item 天皇の背後に元老がいる
\item 天皇の下に三権がある

\item 元老は外から見えないのに、最も天皇の意思決定に影響を及ぼす
\item 枢密院はこっそり何かを決める機関だが、
天皇に失態があった時の身代わりであったという説もある
\item 三権はよく元老などとぶつかっていた
\end{itemize}
\end{itemize}
\end{itemize}

\subsection{諸法典の整備}
「近代国家」であるために

◎最初に整備されたのは刑法(1880)
\begin{itemize}
\item 条約改正に必要だった
\item 刑法が整備されていないと治外法権を解決した時に外国に不利益が出る
\end{itemize}

\bold{1889年} 憲法・皇室典範が最高法規であると定める。
\begin{itemize}
\item 憲法第二条で皇室の後継者などは皇室典範に則ると記載
\item 皇室典範が最高法規でないと誰かが天皇になって暴走するかもしれない
\end{itemize}

\subsection{初期議会}
\subsubsection{選挙制度}
\bold{1889年 \red{衆議院議員選挙法}}
\begin{itemize}
\item 選挙権が与えられる条件
\begin{itemize}
\item 満25歳以上
\item 直接国税を15円以上納めている(主に地租)
\item 男子である(家長である主人がなると見ると差別ではない・・・のかもしれない)

\item 該当者は全人口の1.1\% 程度であった。(課税要件的にほとんどが地主)
\end{itemize}

\end{itemize}

\subsubsection{初の総選挙}
\begin{itemize}
\item 1890年に行われる
\item 投票率93.9\%
\item 勝ったのは旧民権派(地主が投票するので当然)
\begin{itemize}
\item 旧民権派=「\red{民党}」:立憲自由党・立憲改進党
\item 対する「吏党」:\red{超然主義}(外の動きに振り回されるな)の政府支持派
\end{itemize}
\item ただし政府は選挙と関係ないので形式的と言えばそうかもしれない
\end{itemize}

\subsubsection{初期議会}
\bold{1890年} 第1回帝国議会
\begin{itemize}
\item 民党:政費節減・民力休養(地租軽減を要求)
\item 政府:軍備補強→予算増加を望む
\begin{itemize}
\item 主権線(日本の領域の端のライン)及び、
利益線(主権線より外の、軍事的な牽制の及ぶ範囲)の防衛
\end{itemize}
\item 第1から第4までの議会では予算の議論が争点に→民党の一部を切り崩すことに
\end{itemize}

\subsubsection*{第5・第6議会}
\begin{itemize}
\item 争点は条約改正
\item 日清戦争の発生で民党と吏党が歩み寄る
\end{itemize}

◎ここまでで、だいぶ「近代国家」に近づいたよね


\clearpage
\section{藩閥と政党}
\subsection{政党内閣の成立}
初期議会における政府と民党の勢力は
\begin{csvtab}{ccc}
, 政府+軍, 民党
, 超然主義(非政党), 政党
, 藩閥(薩長), 非藩閥→政友会
\end{csvtab}

◎1894-95年 日清戦争が勃発
\begin{itemize}
\item 戦後処理の後、軍拡を継続する(ロシア・欧米対策)
\item 第2次伊藤博文内閣が発足(1892-96)
\begin{itemize}
\item 自由党と提携する(だいぶ丸くなった)
\item 提携の条件として自由党首板垣を内相に任命
\item 当時の内相は現在で言う総務省・官房長官ほか重要な役職をすべて担当していた
\item 伊藤博文はこの人事で少なからず政府から批判を受ける(政党政治のためにはやむなし)
\end{itemize}
\item 第2次松方正義内閣発足(1896-98)
\begin{itemize}
\item \red{進歩党}(立憲改進党の流れをくむ)と提携
\item 大隈重信を外相に任命
\end{itemize}
\item 軍拡のためにお金が必要だが、地租を増徴すると民権派に怒られるので躊躇していた
\item 第3次伊藤博文内閣発足(1898.1-6)
\begin{itemize}
\item 地租増徴案を提案
\item 自由党と進歩党が反対→合体して\red{憲政党}を作る
\item 否決されて解散総選挙(憲政党が勝利)
\item そこまで言うならお前らがやれ
\end{itemize}
\item 第1次大隈重信(憲政党)内閣が発足(1898.6-11)
\begin{itemize}
\item 首相は大隈重信、内相は板垣退助(隈板内閣)
\item 陸海相以外の閣僚が憲政党
\item \red{日本初の政党内閣}
\item 伊藤サイドとしては
\begin{itemize}
\item これ以上世論を無視できない
\item 陸海相のポストを握っていれば最悪なんとかなる
\item どうせ政策担当能力もないだろうしすぐに返ってくる
\end{itemize}
という目論見があった
\item 実際に
\begin{itemize}
\item ポストの奪い合いで内部対立
\item 藩閥勢力からの圧力
\end{itemize}
で憲政党は分裂しておしまい
\end{itemize}
\item 憲政党は分裂した後、
\begin{itemize}
\item 自由党系:新しく別の\red{憲政党}を作る
\item 進歩党系:仕方なく\red{憲政本党}を作る
\end{itemize}
\end{itemize}

\subsection{立憲政友会}
\bold{◎第1次\red{山県有朋}内閣発足(1898-1900)}
\begin{itemize}
\item 山県は「長州」「陸軍」「超然」属性を持っている
\item ついでに政府派の中心人物である
\item 就任してすぐに地租増徴を成し遂げる
\begin{itemize}
\item 2.5\%から3.3\%に増税
\item 旧自由党の支持を得るために
\begin{itemize}
\item 税率を4\%から3.3\%に引き下げ
\item 選挙法を改正:納税15円→10円
\end{itemize}
\end{itemize}
\item 文官任用令(1899)
\begin{itemize}
\item 政党関係者がコネで入庁するのを禁止
\item 政党と政府を切り離す目的
\end{itemize}
\item \bold{\red{軍部大臣現役武官制}(1900)}
\begin{itemize}
\item 軍部大臣は現役の中将または大将でなければいけない
\item 政党の人間が軍に干渉できないようにする(後に活躍)
\item 現在はシビリアンコントロール(文民でなければいけない)
\end{itemize}
\item \bold{\red{治安警察法}(1900)}
\begin{itemize}
\item 19世紀末ごろには労働運動・社会主義運動などが動き出す気色があり、
一度起こると国を揺るがしかねないのでそれを規制して取り締まり
\end{itemize}
\item 旧自由党をたらしこんでおいて反政党政策を立て続けに打ち出す
\begin{itemize}
\item 当然旧自由党は不満を持つ
\item しかし入閣できないので力がない
\end{itemize}
→伊藤博文に接近
\end{itemize}
\bold{◎\red{立憲政友会}を結党(1900)}
\begin{itemize}
\item 総裁:伊藤
\item 中身は政党派の人間と藩閥勢力のうちの伊藤派の人間
\end{itemize}
\bold{「山県系勢力」と対立の構図}になる
\begin{itemize}
\item 「超然」(非政党)系の人間
\item 藩閥勢力のうちの山県派
\item 貴族院・枢密院・軍
\end{itemize}

\bold{◎対立の構図が大きく変化}

\subsection*{}
\bold{◎第4次伊藤内閣(1900-01)}
\begin{itemize}
\item 政友会が基盤
\item 貴族院の抵抗を受けて思い通りに動けない
\end{itemize}

\bold{◎第1次\red{桂太郎}内閣(1901-06)}
\begin{itemize}
\item 桂は「長州」「陸軍」「超然」属性
\item 山県系のNo.2
\end{itemize}

\bold{◎第1次\red{西園寺公望}内閣(1906-08)}
\begin{itemize}
\item 西園寺は「政友会」系で、伊藤の後継者
\end{itemize}

\bold{伊藤・山県は現役を引退し、\red{元老}として影響力を持つ}
\begin{itemize}
\item 元老は徐々に減っていく(死んでいくが、補充されない)
\item 最後の元老は西園寺公望
\item 日本の政治が機能するかは西園寺の政治的バランス感覚に依存していた。
\item 西園寺が元老でなくなるのは\bold{1940年}。
元老亡き後、日本の政治は軍部に乗っ取られていくことになる。
\end{itemize}

\clearpage
\section{条約改正}
\subsection{条約改正交渉の目的}
\bold{◎不平等条約}
\begin{itemize}
\item 片務的MFN待遇
\item 領事裁判権→法権回復
\item 関税自主権の欠如→税権回復
\end{itemize}

税権がお金に関わる問題なので最初にやろうぜ

法権は悪くても腹立つだけだし


\subsection{経過}
1878 寺島宗則が税権回復交渉→失敗

◎同時期に法権に関する問題が多発し、世論の関心が法権に向く
\begin{align*}
\begin{simul}
1882-87 井上馨\\
1888-89 大隈重信
\end{simul}が法権回復交渉するも失敗(イギリスの反対が強い)
\end{align*}

\bold{◎ロシアが極東進出を目論み始める}
\begin{itemize}
\item シベリア鉄道着工(1891)
\end{itemize}
→イギリスが日本に歩みよる姿勢を見せる

\bold{◎1894 陸奥宗光(外相)が\red{日英通商航海条約}を締結}
\begin{itemize}
\item MFNを双務化
\item 領事裁判権を撤廃
\item 関税の引き上げ(協定関税の撤廃までは行かなかった)
\item 5年後に発効(憲法・法体制の整備を待ってから)
\end{itemize}

\bold{◎1911 小村壽太郎(外相)が改正日米通商航海条約を締結}
\begin{itemize}
\item 関税自主権を回復
\item この背景で日本は軍事力を強化していく
\end{itemize}

\clearpage
\section{日清戦争}
\subsection{朝鮮問題}
\subsubsection{開国}
\bold{◎日朝修好条規}→日本は中継貿易で利益を得る
\begin{itemize}
\item[輸入] 米・大豆などを安価に
\item[輸出] イギリス製の綿製品
\end{itemize}
→朝鮮の農村では混乱が起こる

\subsubsection{朝鮮国内の動きと日中}
朝鮮では「李王家」と「閔氏(びんし)」が争っていた。
政権を握っていたのは閔氏だった。
1882年、李王家の大院君(当主の父)が\red{壬午軍乱}(反閔・反日のクーデター)を起こす。
日本は閔氏に兵士を送り、中国も閔氏に加勢して暴動を鎮圧する。

暴動の後、閔氏の中でも保守的な事大党が政権を握る。
1884年、対する独立党が事大党に対して\red{甲申事変}(クーデター)を起こす。
日本は独立党を支援して派兵。
事大党は密かに清に派兵を要請、クーデターを鎮圧する。

その後、1885年に天津条約が締結され、
\begin{itemize}
\item 日清両国の撤兵
\item 派兵時の相互の事前通告
\end{itemize}
を決定した。

\subsection{日清戦争}
\subsubsection{日清戦争}
1889 朝鮮で防穀令が発布される
\begin{itemize}
\item 不作と日本からの穀物の買い占めで国内で食料危機が起こる
\item 日本への穀物の輸出を禁止する
\item 日本から反発を受ける(賠償も要求)
\end{itemize}

\bold{◎1894 \red{甲午農民戦争}}
\begin{itemize}
\item 減税と反日を要求する農民反乱
\item 日清が派兵して反乱自体は停戦
\item この時日本は
\begin{itemize}
\item 十分な軍備
\item 英国が好意的(戦争の容認・英の介入がないことを保証)
\end{itemize}
という戦争をするに十分な状況だった。
\end{itemize}

\bold{→\red{日清戦争}}
\begin{itemize}
\item 朝鮮の独立確保
\item 中国の宗主権を排除

を大義名分に開戦する。
\end{itemize}


日本は(意外にも)優勢で戦を進める。

\subsubsection{\red{下関条約}}
\begin{itemize}
    \item 朝鮮独立の承認
    \item 遼東半島・台湾・澎湖諸島の割譲
    \item 2億両(テール)の賠償金(テールの方が大きい、3億円分)
    \item 長江流域(南清の大河・イギリス勢力圏)の都市の開港

        →MFN(イギリスへの最恵国待遇)で、イギリスが貿易しやすくなる
\end{itemize}

\subsubsection{\red{三国干渉}}
\begin{itemize}
    \item ロシア(シベリア鉄道を作る上で邪魔)
    \item フランス(ロシアと同盟・ロシアに投資もしていて、力がある)
    \item ドイツ(ロシアに攻撃されると嫌なのでロシアの力を東に向けよう)

        (ついでにヨーロッパ各国もあまり日本の中国進出をよく思ってない。)
\end{itemize}
の三国が遼東半島の返還を要求する

→日本は3千万両と引き換えに承諾→軍備に回す(これで後にロシアと戦う)

(3千万両は清が払う。この借金を列強が肩代わりする形で中国は乗っ取られていく)

\subsubsection{台湾支配}
1895年に台湾総督府設置(植民地化の象徴)

\clearpage
\section{日露戦争}
(急に"韓国"という名前が出てくるが、1897年から朝鮮は大韓帝国となり、それを指している。)
\subsection*{大局}
\begin{description}
    \item[日本]韓国を利益線に入れて満州の方まで力を示したい
    \item[ロシア]三国干渉(露仏同盟)でフランス・ドイツと結んで満州を占領している。
        イギリスの影響力を取り除きたい。
    \item[ドイツ]帝国主義に乗り遅れたオーストリア・イタリアと三国同盟を結んでいる。
        今はロシアと手を組んで、イギリスの影響を取り除いて中国の利権を手に入れたい。
    \item[イギリス]中国の利権を独占したい。
        日本よりも独露仏の方が怖いので、日本をうまく使おうとしている。
\end{description}

\subsection{中国分割}
日清戦争によって中国の弱体化が露呈した。
それにより、今まで暗黙的に行われていた「\red{中国分割}」が本格化する。
\begin{itemize}
    \item 租借地(中国から土地を借り受ける)
    \item 鉄道敷設権(鉄道を敷いて、その地域に自国の色を付ける)
    \item 鉱山・工場の経営権(収益を吸い取る)
\end{itemize}
このサイクルで中国は半植民地化していく。

ロシアは遼東半島に鉄道を通して、旅順・大連を租借、
ウラジオストクまでの直通路を築く。

◎これに対して、中国各地で民衆による反侵略闘争が起こる。

\bold{1900年 \red{義和団事件}}
\begin{itemize}
    \item 北京を包囲して各国の大使館を制圧、人質をとる
    \item 清政府がこれに乗じて各国に宣戦布告、条件交渉に乗り出す
    \item 日露を中心とする8カ国の列強連合軍が北京を占領(北清事変)
        →首都を制圧されて半ば敗戦国状態に
\end{itemize}

◎\bold{1901年 \red{北京議定書}}
\begin{itemize}
    \item 4億5千万両の賠償金
    \item 列強軍の北京駐留
\end{itemize}

\clearpage
\section{日英同盟}
\begin{description}
    \item[ロシア]:旅順・大連を租借して、鉄道を通している

        北清事変で\bold{満州を占領}
    \item[日本]:韓国の支配を目指す
    \item[イギリス]:中国利権の確保を目指す
\end{description}
ロシアの満州占領は日英両国の脅威となる。

\subsubsection*{\red{日露協商論}(伊藤・井上など)と\red{日英同盟論}(山県・桂・小村など)}
日露協商論:満韓交換論(満州の利権はロシアに認めて、韓国の利権は認めてもらおう)

◎1901年 第4次伊藤内閣から第1次桂内閣へ
→日英同盟論に踏み切り

◎\bold{1902年 \red{日英同盟(協約)}}
\begin{itemize}
    \item イギリスの中国利権と、日本の韓国利権を相互に承認する
    \item 他の国と戦争になったら
        \begin{itemize}
            \item 相手国に加勢なし→中立
            \item 相手国に加勢あり→加勢
        \end{itemize}
\end{itemize}

この時、イギリスはアフリカの侵略とインドの制圧で手一杯だったので、
日本を中国向けの軍隊として使おうとしている。

\subsection{日露戦争}
日英同盟の後もロシアとは交渉を続けるも難航
→国内の反露感情が「主戦論」となって世論の主流となる。

※アメリカに背後から攻撃されると困る
\begin{description}
    \item アメリカが中国に入りたいと「門戸開放宣言」を投げる
    \item ロシアの除去は門戸開放のためだとアメリカをそそのかす
    \item アメリカが協力を約束(戦費の提供)
\end{description}

→\bold{1904.2 対露宣戦}→日本が「優勢」
\begin{itemize}
    \item ロシアの常套戦術「退却戦」はうまくいっており、ハルビンまで日本軍を誘引していた
        →ハルビンを包囲攻撃すればロシアが勝ってたという見込みもあり
    \item ロシアの誤算はバルチック艦隊の敗北
        (イギリスの協力でバルチック艦隊は資材が不足している状態だった)
    \item ロシア国内で反体制運動→鎮圧しないと中から潰れる
    \item ただし日本も疲弊(戦費は借金、物資も不足)
\end{itemize}
→米国(セオドア=ルーズベルト)に調停を依頼

◎1905.6 和解勧告→\bold{1905.9 \red{ポーツマス条約}}
\begin{itemize}
    \item 日本の韓国に対する指導・監督権を承認
    \item 旅順・大連の租借権及び長春から旅順間の鉄道利権を日本に移譲
    \item 南樺太の割譲
        など
\end{itemize}

◎国内への影響
\begin{itemize}
    \item 戦死者・戦費が日清戦争よりも桁違いに多い
    \item 賠償金なし→戦費は国民の負担→国民の不満
    \item \red{日比谷焼き討ち事件}(小村の家の近辺を攻撃)

        日本では珍しい暴動
\end{itemize}
\end{document}

