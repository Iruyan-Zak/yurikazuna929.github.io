\documentclass[12pt,fleqn]{ltjsarticle}
\usepackage{mymacro}
\usepackage{notemode}
\usepackage{mathmode}
\begin{document}

\section{明治維新}
\subsection{新政府の基本方針が発表されます!}

\bold{1868.3.14 \red{五箇条の誓文}}
\begin{itemize}
\item \red{公議世論} (みんなの意見)の尊重 => (諸大名)
\begin{itemize}
\item 諸大名の話し合いで決まると思っていたが、そうではないらしい
\item むしろ、条文には大名をなくしていく狙いがあった(事実、なくなった)
\end{itemize}
 \item \red{開国和親} (国際法を守ります) => (諸外国)
\item 政体書(政府組織)・・・三権分立っぽくしている
\item 明治改元・\red{一世一元の制}
\item 東京遷都(大坂に遷都する案もあった)
\end{itemize}

\subsection{中央集権体制の構築}
\begin{itemize}
\item 旧幕府領 → 新政府領(府・県)
\item 親藩 → そのまま
\end{itemize}
これを\red{府藩県三治制}という。
\begin{itemize}
\item ただしこれでも新政府の財政基盤は以前と同じなので、もうちょっとお金が入る仕組みが欲しい。
\end{itemize}
\subsubsection*{1869年 \red{版籍奉還}}
\begin{itemize}
\item 版=領地
\item 籍=領民
\item 奉還=朝廷に返還する
\item 藩主の仕事は\red{知藩事}に移った(藩政は維持されている)
\begin{itemize}
\item 知藩事には中央政府の役人が就任した
\item もともとの藩主をおいておくと危険なので、お金(家禄)を与えて藩から追放した
\end{itemize}
\end{itemize}

Q.なぜ大きな反乱が起こらなかったか?

A.
\begin{itemize}
\item 最大の大名家だった徳川家が簡単に潰されたから
\item お金がなくて反乱の余裕がなかったから
\item お金がある藩が改革を先導していたから
\end{itemize}

\subsubsection*{1871年 \red{廃藩置県}}
準備として御親兵(薩長土の藩兵1万人)を配備しておいた。
\begin{itemize}
\item 藩を全廃して、代わりに県・府をおいた
\item 知藩事を解任して東京に召集、代わりに中央から\red{府知事}・\red{県令}を派遣した。
\item もともとの藩主からの抵抗は小さかった
\end{itemize}
\subsection{中央官制の改革}
\begin{itemize}
\item 太政官制:七官制→二官六省→三院制(八省)
\item \red{藩閥政府}:薩長土肥出身者が要職を独占した
\end{itemize}

\section{明治国家の形成}
「さてさて、『国民』を作りましょうか」・・・四民平等
\subsection{身分制の解体}
\begin{itemize}
\item 大名・公家 → 華族
\item 藩士・幕臣 → 士族
\item 農工商   → 平民

\begin{itemize}
\item 平民の苗字使用許可(後に強制)

戸籍単位での国民の管理が可能に
\item 身分間の通婚許可
\item 移動・職業の自由
\item 問題点:華族・士族には依然として秩禄(家禄と賞典禄)を支給している

→財政が圧迫される

これらから特権を奪い取ることで出費を減らしていくお話
\end{itemize}
\end{itemize}

\subsection{秩禄処分}
\subsubsection*{1873年 \red{秩禄奉還の法}}
希望者に一時金を渡して、以降の秩禄を放棄してもらう
\subsubsection*{1876年 \red{全禄公債証書}を発行}
\begin{itemize}
\item 家禄の全廃(\red{秩禄処分})に成功
\item 5年から14年分の証券で、すぐにお金がもらえるわけではなかった

→(騙された)武士は商売などをしないと生活できない状況になった。
\begin{itemize}
\item 軍人・警察官・官僚・学者・技術者・教師・実業者など
\item 事業に失敗して没落する者もいたが、成功者も少なくなかった。
\end{itemize}
\end{itemize}

\subsection{国民皆兵}
ヨーロッパの軍事制度を参考に新政府を支える常備軍を創設しましょう!

\subsubsection*{1873年 \red{徴兵令}}
\begin{itemize}
\item 満20歳以上の男子が3年間兵役につく
\item ただし、免役規程は多かった(一人っ子は免除など)
\item 兵役を嫌がる人が多かった
\item 宣伝文句「血税を捧げよ」が元で\red{血税一揆}が起こった
\item これにより「武士」の存在意義がなくなった
\end{itemize}

\subsection{外交1}
\subsubsection{岩倉使節団}
<目的>
\begin{itemize}
\item 新政府の「開国和親」を宣伝する
\item 条約改正の予備交渉(失敗)
\item 欧米視察(大成功)→近代化の必要性を再認識
\end{itemize}
<メンバー>
\begin{itemize}
\item 岩倉具視
\item 木戸孝允
\item 大久保利通
\item 伊藤博文

など
\end{itemize}
大久保利通は殖産興業政策を視察しまくった。

\subsubsection{留守政府}
<メンバー>
\begin{itemize}
\item 三条実美
\item 大隈重信
\item 西郷隆盛
\item 板垣退助

など
\end{itemize}
<政策>
\begin{itemize}
\item 四民平等
\item 徴兵令
\item 学制
\end{itemize}
留守政府はなかなか帰ってこない使節団に腹を立てていた。

\subsubsection{征韓論}
朝鮮とは鎖国して以来、国交を拒否され続けていた。

\red{征韓論}:\bold{武力行使}を視野に入れた開国交渉を目指す動き。

開国論は政府だけでなく、(武士の)世論でもあった。

\begin{itemize}
\item 強硬論:留守政府(西郷・板垣など)が主導
\item 内治優先論:帰国した使節団(大久保・木戸など)が主導

国内の政治・経済の整備を優先するべきだと考えた。
\end{itemize}

意見は対立、両者の溝は深まっていった。

\begin{itemize}
\item 先に、留守政府が西郷を朝鮮に派遣することを決定した
\item 内治派の工作によって撤回させられる
\item 征韓派(西郷・板垣・江藤など)が参議を辞任する
\end{itemize}

\bold{1873年 \red{明治六年の政変}}
\begin{itemize}
\item 大久保の独裁的な政治が始まる
\item これを機に日本の殖産興業が発展していく
\end{itemize}

\subsection{士族の反乱}
◎明治維新以来、「武士」の解体が進んでおり、
特権を失った武士は「不平士族」へと成り下がっていた。

→はじめに武力による反政府活動を行う
\begin{itemize}
\item 1874年 江藤新平による佐賀の乱
\item 1876年 秋月の乱・萩の乱など
\item 1877年 2-9月 西郷隆盛による\red{西南戦争}

西郷はこの戦いで戦死する。
\end{itemize}

西南戦争が完全に鎮圧されて以来、不平士族は武力による反乱では政府に勝てないことを悟る。

→言論による反政府運動へ

\subsection{外交2}
実は日本は植民地化の危機にあった(中国分割と同じ状況にあった)

\begin{itemize}
\item 幕府・藩の対外負債→1875年までに完済
\item 鉄道や工場の利権→1873年までにすべて買い戻し
\end{itemize}
これには莫大な金がかかったが、これができなかった中国は植民地化したことを考えると
この行動は日本のファインプレーであったと言える。

\subsubsection{対中国}
\bold{1871年 \red{日清修好条規}}
\begin{itemize}
\item 相互に開港
\item 相互に領事裁判権
\item つまり対等条約である(これには対朝鮮外交を有利にする意図があった)

朝鮮は中国に守ってもらっていたので、中国と対等になれば朝鮮の義兄みたいな感じになれるかも?
\end{itemize}

◎明治時代初頭まで、琉球は日清に両属し、日本も薩摩もそれを容認していた。
(というより、琉球を通して中国と貿易をしていたので逆に助かっていた)

1871年 琉球藩を設置する
\begin{itemize}
\item 国内では廃藩置県が起こっていたが、琉球は自治区として特例
\item 国王である尚泰は藩王に格下げ
\end{itemize}
1871年 琉球漁民殺害事件
\begin{itemize}
\item 琉球漁民が台湾で殺害されたのに乗じて、「我が国民に何をする」と中国に肉薄
\item 中国は「台湾人はうちの国民じゃないので・・・」と回避
\item 台湾が中国の傘下に入っていないことを確認
\end{itemize}
1874年 台湾出兵
\begin{itemize}
\item 中国はやっぱり気分が悪いので「お金払うから帰ってくれない?」と言う
\item 日本はお金をもらって帰った
\item 対外派兵は不平士族の不満を和らげるための施策でもあった
\end{itemize}
1879年 沖縄県を設置する
\begin{itemize}
\item 尚泰は東京に移住させられる
\end{itemize}

\subsubsection{対朝鮮}
\bold{1875年 \red{江華島事件}}
\begin{itemize}
\item 対朝鮮版「黒船来航」
\item 船を近づけて攻撃してきたところに謝罪を要求する外道技
\item 黒船と同じノリで条約締結

\bold{1876年 \red{日朝修好条規}}
\begin{itemize}
\item 釜山・仁川・元山を開港
\item 日本の片務的領事裁判権
\item 日本の片務的無関税権
\item ゴリゴリの不平等条約
\end{itemize}
\end{itemize}

\subsubsection{国境の画定}
\bold{1875年 \red{樺太千島交換条約}}
\begin{itemize}
\item ロシアに樺太を渡して千島を受け取る
\item 領域的には得はしないけど樺太の管理が面倒だったのでOK
\item 千島列島の漁業権も受け取る
\end{itemize}
 
1876年 小笠原諸島の領有を宣言

\subsection{殖産興業}
\subsubsection{通貨・金融制度}
\bold{1871年 \red{新貨条例}}
\begin{itemize}
\item 十進法の通貨単位「円・銭・厘」を導入
\item 1円は1両と同じ価値(=金の重さを基準にした通貨単位である)
\end{itemize}
\bold{1872年 \red{国立銀行条例}}
\begin{itemize}
\item 民間銀行が生まれる
\item 国立銀行ならどこでもお札(国立銀行券)刷っていいよ
\begin{itemize}
\item 正貨兌換義務あり
(銀行券をいつでも本物の金に交換できるように金を準備しておけ)

この時は不換紙幣ではなかった
\item そのせいで銀行設立は進まなかった(4行(第一\~第四)のみ)
\item 1876年に金兌換義務を廃止→1879年までに153行に増加
\end{itemize}
\end{itemize}
※銀行の単位「行」:よみかた「こう」

\subsubsection{地租改正}
幕藩時代以来の税制も変えちゃおうよ

◎新政府は幕府・藩の債務を継承しており、財政は不安定だった。
しかも、歳入は年貢が基本だった。

年貢の決め方は石高によるものだった。
石高は一言で言うと土地の「良さ」である。
「良さ」を直接お金で表したものが地価である。
土地に値段がつくということは、土地を売り買いできるようになる。

ここに、封建的土地制度が解体された。
\begin{itemize}
\item 1871年 田畑勝手作を許可(用途の自由化)
\item 1872年 田畑永代売買禁令(売買の自由化)
\item \bold{1873年 \red{地租改正条例}}
\begin{itemize}
\item 課税基準:収穫高→地価
\item 納税手段:物納 →金納
\item 納税者 :耕作者→土地所有者
\end{itemize}
税額は法定地価の3\%。この値は政府の歳入が減らないように決めた値である。
\end{itemize}

→地租改正事業(土地調査・明治版検地)→1880年ごろに完了

地価と所有者が決まったら\red{地券}が発行された。

<影響>
\begin{itemize}
\item 近代的税制の確立→新政府は税収が安定した
\item 土地が資産になる→農村では貧富の格差が増大
\begin{itemize}
\item 小作農→貧困に苦しんで出稼ぎ(工業労働者)へ
\item 自作農→土地を売って小作農に
\item 地主 →土地を集めて大規模に
\end{itemize}
\item 課税量は変わっていない→農民の税負担は軽減されなかった
\end{itemize}
→1876年ごろを中心に地租改正反対一揆(同時期に士族も反乱)

農民と士族が手を結ぶと面倒なことになる

→1877年 地租税率を2.5\%に引き下げ


\subsubsection{殖産興業}
担当官庁
\begin{itemize}
\item 工部省(1870年:伊藤博文)
\item 内務省(1873年:大久保)
\end{itemize}
軽工業(繊維工業)・農業・海運・鉱工業(造船・兵器・鉱山・鉄道など)
の発展・近代化
\begin{itemize}
\item 幕府・藩の官営化
\item 交通・鉄道
\begin{itemize}
\item 新橋\~ 横浜(1872年)
\item 神戸\~ 大阪\~ 京都(1874年)
\item イギリスの支援を受けたが、利権は渡さなかった
\end{itemize}
\item 通信・電信
\begin{itemize}
\item 東京\~ 横浜(1869年)
\item 東京\~ 長崎(1873年)
\item 長崎には外国からの電信が入るので、
実質明治6年には世界と交信出来ていた。
\end{itemize}
\end{itemize}
○官営模範工場・・・技術導入・教育

スタイル:(官営事業・学校)+お雇い外国人
\begin{itemize}
\item 外国人を高給で雇ったが、授業そのものは渡さなかった
\item むしろ、早い段階で日本人技術者を作って講師にすることに成功している
\item 外資の侵入を抑え、植民地化を警戒した(鉄道・鉱山を守る)
\end{itemize}
海運:三菱を優遇

\subsection{文明開化}
\subsubsection{教育}
\bold{1871年 \red{文部省}を創設}

\bold{1872年 \red{学制}を公布}
\begin{itemize}
\item 小学校の義務教育化
\item 国民皆学(身分問わず)を目標にしていた
\item しかし当初、就学率は3割程度にとどまっていた
\begin{itemize}
\item 子供を小学校に行かせると働き手がいなくなる
\item 授業料を取られるので通わせられない
\item 学制反対一揆も起こった
\item そのうち、小学校に通わせたほうがいいことがわかってきだして、
20世紀には就学率はほぼ100%になった。
\end{itemize}
\end{itemize}

\subsubsection{思想・宗教}
◎近代思想が導入されます。
\begin{itemize}
\item 福沢諭吉などが中心の\red{明六社}が明六雑誌などを出版して啓蒙活動を行った。
\item 新聞が作られる
\end{itemize}
→ジャーナリズムや民権運動につながっていく。

◎宗教政策

政府は「祭政一致」の考えのもと、神道を国教化した。

しかしそれまでは「神仏習合」の考えがあり、神社も寺も一緒に建てられていた。

→\bold{1868年 \red{神仏分離令}}
\begin{itemize}
\item \red{廃仏毀釈}

寺を破壊し、仏具を捨て、仏教を排除していった。
\end{itemize}

\section{自由民権運動}
\subsection{背景}
◎士族の不満
\begin{itemize}
\item 士族は自ら明治維新を推め、武士を解体していった
\item 反政府運動
\begin{itemize}
\item 武力:士族反乱
\item 言論:自由民権運動
\end{itemize}
\end{itemize}
明治六年の政変以降、政治家がこれらの運動に参加し、指導していった。


◎農民の不満
\begin{itemize}
\item 血税一揆
\item 地租改正反対一揆
\item 学制反対一揆
\end{itemize}

\subsection{自由民権運動}
\subsubsection{運動の開始}
◎政社の結成
\begin{itemize}
\item \bold{1874年 \red{愛国公党}を結成}:板垣退助・後藤象二郎ら

→\red{民撰議院設立建白書}
\item \bold{1874年 \red{立志社}}(高知):片岡健吉・板垣ら
\item \bold{1875年 \red{愛国社}}(大阪):立志社が全国組織を目指して設立
\end{itemize}
◎政府の対応
\begin{itemize}
\item \bold{1875年 \red{大阪会議}}:大久保・板垣・木戸など
\item 明治天皇が\red{漸次立憲政体樹立の詔}を発する
\item 立法機関として「元老院(上院)」、「地方官会(下院)」、
司法機関として「大審院」を設立
\item 言論統制を強化
\begin{itemize}
\item 1875年 讒謗律(政府の文句を紙面に乗せるとOUT)
\item 新聞紙条例
\end{itemize}
\end{itemize}

\subsubsection{運動の進展}
\bold{1877年 \red{立志社建白}}
\begin{itemize}
\item 国会開設
\item 地租軽減
\item 条約改正
\end{itemize}
これは政府に却下される。

この時、運動の支持者は士族から地主や商工業者にまで拡大していた。

\bold{1880年 \red{国会期成同盟}}(不受理)

これに対して政府は、集会を規制する\red{集会条例}を制定した。

\subsection{明治十四年の政変}
1878年。大久保が暗殺されてしまいました。

\begin{csvtab}{ccc}
候補, 国会開設の時期,立憲政治のモデル
大隈重信,早期(2年後),イギリス(政党政治)
伊藤博文,漸進的,プロイセン
\end{csvtab}
多数派であった伊藤博文が後継する。

\bold{1881年 \red{開拓使官有物払い下げ事件}}
\begin{itemize}
\item 開拓使(役所)を閉める際にお友達に官有物を
安く売り払った(えこひいきした)事件
\item 干ばつも相俟って民権派(世論)の批判が殺到
\item 大隈が民権派の肩を持つ
\item 政府の中で大隈が民権派を扇動したという噂が立つ
\item 伊藤ら多数派、天皇の力で大隈参議を罷免する
(\bold{\red{明治十四年の政変}})
\item とりあえず文句を言われないように払い下げは中止
\item \red{国会開設の勅諭[詔]}を出して1890年の国会開設を約束
\end{itemize}

\subsection{松方財政}
\subsubsection{背景}
\begin{align*}
&\begin{simul}
明治維新諸政策\\
西南戦争(1年分の予算を使用)
\end{simul}\rightarrow 政府は資金難になる\\
&\Rightarrow
\begin{simul}
国立銀行からの借り入れ(国立銀行券を発行して対応)\\
政府が自分たちで紙幣を発行
\end{simul}= 不換紙幣が大量に刷られた\\
&\begin{simul}
\Rightarrow {\rm \bold{インフレ}}\\
対して地租は定額
\end{simul}\rightarrow 政府はさらなる財政難に陥る。
\end{align*}

\subsubsection{松方財政}
1881年ごろ 大蔵卿 大隈重信の政策
\begin{itemize}
\item 増税:酒造税など、嗜好品に対して増税する
\item 官営事業(必須・かつ赤字)の払い下げに着手
\end{itemize}
→明治十四年の政変:大隈罷免

\bold{1881年 大蔵卿\red{松方正義}が政策を継承}
\begin{itemize}
\item 軍事費以外の歳出抑制
\item 増税:酒・たばこなど
\item 官営事業の払い下げを継続
\end{itemize}
$\rightarrow$余剰金(利益・黒字)が生まれる
\begin{align*}
\Rightarrow\begin{simul}
借入金返済(国立銀行側で償却)\\
政府紙幣償却
\end{simul}\rightarrow
インフレが落ち着くと同時にデフレに突入する
\end{align*}

◎政府紙幣償却について
\begin{itemize}
\item 政府は紙幣を発行しました = 神様からお金を借りました
\item 政府に黒字が生まれました = 返済する用意が出来ました
\item 政府は自分で発行した紙幣を処分しました = 神様にお金を返しました
\end{itemize}

\bold{1882年 \red{日本銀行}を設立}
\begin{itemize}
\item 国立銀行から国立銀行券を発行する機能を取り上げる
\item 代わりに銀兌換の日本銀行券を発券する機能を持っている
\end{itemize}

\subsubsection{影響}
デフレーション→不況
\begin{itemize}
\item インフレ時には景気が良くなる場合と
悪くなる場合(スタグフレーション)があり得る
\item しかしデフレ時に景気が良くなることはない
\begin{itemize}
\item 物価が継続的に下落する状況では、必要なものでない限りは
後で買うほうが得になる→消費低迷
\end{itemize}
\end{itemize}

これにより中小自作農の生活に打撃(お金は減るのに地租は変わらず)
\end{document}

