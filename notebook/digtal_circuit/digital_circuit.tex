\documentclass[12pt]{ltjsarticle}
\usepackage[top=15truemm,bottom=15truemm,left=10truemm,right=10truemm]{geometry}
\usepackage{fontspec}
\usepackage{luatexja-fontspec}
\usepackage[ipaex]{luatexja-preset}
\setmainfont[Ligatures=TeX]{TimesNewRoman}
\usepackage{amsmath}
\usepackage{empheq}
\usepackage{color}
%\usepackage{ascmac}
\begin{document}
\newcommand{\red}[1]{\textcolor{red}{#1}}
\renewcommand{\labelenumi}{(\arabic{enumi})}

\section{ディジタルIC}
\subsection{ICの規格}
\subsubsection{絶対最大定格と推奨動作条件}
ACファミリの場合、
\begin{itemize}
\item 絶対最大定格:-0.5~7[V]
\item 推奨動作条件:2~6[V]
\end{itemize}
と定義されている。

\red{絶対最大定格}の範囲外の電圧をかけるなどすると、高確率でICは破壊される。

\red{推奨動作条件}の範囲外の電圧をかけるなどすると、ICが安定に動作することは保証されない。

\subsubsection{スイッチング特性}
理想的な入出力信号はその値が変化する時、その変化は直角に表記される。
しかし、実際にはラグがあり、信号の波形は斜めになる。

例えば入出力が0V→5Vになる時、その電圧が0.5V→4.5Vになるのにかかった時間を\red{立上り時間$t_r$}といい、その逆を\red{立下り時間$t_f$}という。

また、ICに入力信号を与えた時、それに対応する出力信号が得られるまでの時間を\red{伝搬遅延時間}といい、この電気的特性を\red{スイッチング特性}という。

\subsubsection{論理レベル}
論理信号(0,1)と実際の電圧との対応を\red{論理レベル}という。

この時、実際の電圧を「0」と評価するか「1」と評価するかの境界を\red{スレッショルド(Threshold)電圧}または\red{閾値電圧}という。
スレッショルド電圧ぎりぎりの入出力を行うのは望ましくない(回路の状態、ICの個体差などによって誤動作する)。そのため、ICではスレッショルド電圧付近の電圧を扱う際の挙動を定義していない。
なお、スレッショルド電圧は、ICのファミリによって異なるので、使用するICのファミリを揃えておくことが重要である。

\subsubsection{プルアップ・プルダウン抵抗}
電源にもグラウンドにも繋がっていない部分はどれだけの電圧を持っているかわからない。
そのため、素子にスイッチから入力を取る場合など、\red{ハイインピーダンス状態}に
なることを避けるために\red{プルダウン抵抗}をスイッチの後に取り付ける。
これにより、スイッチがOFFの時に入力が0になることが保証される。
また、電圧を引き上げるために使うのが\red{プルアップ抵抗}である。

実際、プルアップもプルダウンも同じ働きである。(抵抗とスイッチの場所くらいしか違いがない)

\subsubsection{ファンアウト・ファンイン}
\red{ファンアウト}は、ある素子の出力ピンにつないでいい素子の限界数である。

\red{ファンイン}は、ある素子の入力ピンにつないでいい素子の限界数である。
\begin{itemize}
\item ファンアウト「0」は出力が0の時に素子が出力ピンから吸い込める電流の限界を意味する。
\item ファンアウト「1」は出力が1の時に素子が出力ピンから吐き出す電流の限界を意味する。
\item ファンイン「0」は入力が0の時に素子が入力ピンから吐き出す電流の限界を意味する。
\item ファンイン「1」は入力が1の時に素子が入力ピンから吸い込む電流の限界を意味する。
\end{itemize}
$\sum$ファンイン「0」>ファンアウト「0」の時はまずい。

$\sum$ファンイン「1」>ファンアウト「1」もまずい。

壊れないかもしれないが、ちゃんと「1,0」を判断できなくなるみたいだ。

\subsubsection{オープンドレイン(コレクタ)}
オープンドレインとは、C-MOSのFETのドレイン、もしくはTTLのトランジスタのコレクタを
直接出力ピンにする方式。
コレクタってことは、入力。入力ピンを出力ピンに持ってくる・・・?

オープンドレイン形の正しい使い方は、出力ピンにプルアップ抵抗をつなぐこと。
オープンドレインの利点は「大きな電流を吸い込める」こと。
普通のICでは5Vしか出力できないけど、12V使いたいって時に重宝する。
プルアップ+12Vを出力ピンにつなぐ。
出力ピンを閉じれば出力ピンは12Vになるし、開くと0Vになる。
開くと大きめの電流がIC内に流れるが、所詮エミッタに返すだけなので余裕。

オープンドレインを使えばIC内部の構成によらず、出力ピンに直接つないだトランジスタとかFETの性能だけでファンアウトを決定できる。良さ。

でも、プルアップ抵抗忘れないでね。

あと、出力ピンを短絡した回路(\red{ワイヤード回路})を組むこともできる。
その場合でも、出力が「0」の時に外側の電流をすべて吸い込んでしまうことに注意。
(NAND2つでワイヤード回路を組むとANDになります。。。)


\section{ディジタル回路}
\subsection{コンパレータ}
\red{コンパレータ}とは入力データの大小を比較する回路で、\red{比較回路}とも呼ばれる。コンパレータには通常、{\bf EX-NOR}が用いられるが、EX-ORを使う場合もあり、この場合は\red{一致回路}と呼ばれる。

\subsection{エンコーダ・デコーダ}
\red{エンコーダ}は\red{符号器}と訳され、その意味はコンテキストによって大きく変わってくるが、この文脈では10進数を2進数に変換する回路とする。

\red{デコーダ}は\red{解読器}と訳され、その意味はたいていエンコーダの反対である。

さて、エンコードするには電子回路に10進数を扱わせる必要があるが、
一般的に強引に解決する。
まず、10本の入力ピンと4本の出力ピンを用意する。
ここで、入力は1つのみが「1」で、残りが「0」であると保証することで
擬似的にエンコーダは任意の1桁の10進数の入力を感知できる。

後は適当にORで対応するビットに出力すれば、
10進数が2進化10進数に変換できることになる。

デコードは上の反対である。
注意すべきは4本の入力ピンからNOTのラインを作っておく必要があることである。
あとは4入力ANDを10個適当に並べれば2進化10進数を10進数に変換できる。
\end{document}
