\documentclass[14pt]{ltjsarticle}
\usepackage[top=15truemm,bottom=15truemm,left=10truemm,right=10truemm]{geometry}
\usepackage{luatexja-fontspec}
\usepackage[ipaex]{luatexja-preset}
\usepackage{amsmath}
\usepackage{empheq}
\usepackage{color}
\usepackage{ascmac}
\usepackage{mymacro}
\newdimen\tbaselineshift
\begin{document}
\newcommand{\red}[1]{\textcolor{red}{#1}}
\newcommand{\bold}[1]{{\bf {#1}}}
\renewcommand{\labelenumi}{(\arabic{enumi})}
\section{正規表現}
\bold{記号列(文字列・語)}:アルファベットV(記号の有限集合)から取り出した記号からなる「列」。

\red{空列}:長さ0の記号列。$\phi$や$\lambda$、$\varepsilon$と表記する。

Vから作られる記号列をすべて含む集合をV*(ヴィ・スター)と書く。
\begin{align*}
{\rm V}&=\{a,b\}のとき、\\
{\rm V*}&=\{\phi, a, b, aa, ab, ba, bb, aaa, \dots\}
\end{align*}

\begin{itembox}[l]{注意}
\begin{itemize}
\item V*には空列が含まれる。
\item V*の要素は中かっこで囲わない(集合ではなく文字列)
\end{itemize}
\end{itembox}

\subsection{正規表現(regular expression)}
記号列の集合(V*)を表記する方法の1つ。
[A-Z][0-9]+[a-z]*[1-5|9]?
この正規表現には"E1333akashi9"などが該当する。


\begin{itembox}[l]{正規表現とオートマトン}
\begin{itemize}
\item 正規表現があれば、それに対応する有限オートマトンがある。
\item 有限オートマトンがあれば、それに対応する正規表現がある。
\end{itemize}
\end{itembox}

\section{形式言語}
言語には「自然言語(日本語など)」と「形式言語(C言語など)」がある。

\bold{生成文法}:形式言語を生成する規則

\bold{オートマトン}:形式言語を認識する状態機械

\bold{BNF(バックス・ナウア記法)}
\csvtab{ll}{
$\langle hoge\rangle$, 超変数,
::=, (左)を(右)と定義するi,
|, または,
[fuga], fugaはあってもなくても,
}
\end{document}
