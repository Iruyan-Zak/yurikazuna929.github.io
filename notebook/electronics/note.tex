\documentclass[12pt]{ltjsarticle}
\usepackage[top=15truemm,bottom=15truemm,left=10truemm,right=10truemm]{geometry}
\usepackage{luatexja-fontspec}
\usepackage[ipaex]{luatexja-preset}
%\setmainfont{Times New Roman}
\usepackage{amsmath,amssymb}
\usepackage{empheq}
\usepackage{color}
\usepackage{bm}
\usepackage{ascmac}
\newdimen\tbaselineshift

\begin{document}
\newcommand{\red}[1]{\textcolor{red}{#1}}
\renewcommand{\labelenumi}{(\arabic{enumi})}
\newcommand{\grad}[0]{\mathrm{grad}}
\renewcommand{\div}[0]{\mathrm{div}}
\newcommand{\round}[2]{\frac{\partial {#1}}{\partial {#2}}}
\renewcommand{\epsilon}{\varepsilon}
\section{電位分布と電場}
\subsection{電位と電場の関係}
電場$\vec E(\bm E)$電位$V(\bm r)$の時、
\begin{align*}
\bm E&=-\nabla V(\bm r)\\
&=\grad V(\bm r)
\end{align*}
grad:\red{勾配}

\newcommand{\vi}[0]{\bm i}
\newcommand{\vj}[0]{\bm j}
\newcommand{\vk}[0]{\bm k}
\newcommand{\vE}[0]{\bm E}

\begin{align*}
\grad \equiv \nabla &\equiv
(\round{}x,\round{}y,\round{}z)\\
&=(\partial_x,\partial_y,\partial_z)\\
&=\round{}x\vi+\round{}y\vj+\round{}z\vk\\
\end{align*}
$\vi,\vj,\vk$:単位ベクトル
\begin{align*}
|\vi|=|\vj|=|\vk|=1\\
\vi\cdot\vj=\vj\cdot\vk=\vk\cdot\vi=0\\
\vi=(1,0,0),\vj=(0,1,0),\vk=(0,0,1)
\end{align*}
$\bm E=(E_x,E_y,E_z)$を$x,y,z$の成分表示すると、
\begin{align*}
E_x=-\round Vx,E_y=-\round Vy,E_z=-\round Vz
\end{align*}
$V$:スカラー $\vE$:ベクトル

\subsection{電位分布を求めるための基礎方程式}
電荷が連続的に分布する球体(\red{空間電荷})を考える。

この時、単位体積あたりの電荷量\rho[C/m^3]を\red{空間電荷密度}という。

半径rの球体の全電荷量Qは
\begin{align*}
Q=\frac43\pi r^3\rho
\end{align*}
となる。

\subsection*{ガウスの法則}
正の電荷からは電気力線が湧き出し、負の電荷へは電気力線が吸い込まれる。
電気力線の束を\red{電束}といい、
1[C]の電荷からは1本の電束が出ているものと定義する。

微小体積を貫く電束を考える。
ここでは簡単にするために$x$方向に貫く電束のみを考える。

$x$方向の電場の変化量は
\begin{align*}
E_x\rightarrow E_x+\round{E_x}{x}dx
\end{align*}
これは微小体積での$x$方向の電場の変化量の変化を表す。

\begin{align*}
(電束の変化)&=\epsilon_0\round{E_x}{x}dxdydz\\
\\
\epsilon_0\round{E_x}{x}dx&:電束密度\\
dydz&:貫く面積\\
\epsilon_0&:真空の誘電率 [{\rm C^2/(N\cdot m^2)}]\\
\\
&=\rho dxdydz(全電荷)
\end{align*}

$y,z$方向も考えると、
\begin{align*}
&\epsilon_0\left[
\epsilon_0\round{E_x}{x}dxdydz+
\epsilon_0\round{E_y}{y}dxdydz+
\epsilon_0\round{E_z}{z}dxdydz
\right]\\
=&\epsilon_0\left[\round{E_x}{x}+\round{E_y}{y}+\round{E_z}{z}\right]dxdydz\\
=&\rho dxdydz
\end{align*}
よって、ガウスの法則

\begin{align*}
\div{\vE}=&\frac{\rho}{\epsilon_0}(=\nabla\cdot\vE)\\
\div=&(\round{}{x}+\round{}{y}+\round{}{z})
=\round{\vi}{x}+\round{\vj}{x}+\round{\vk}{x}\\
\vE=&-\grad{V}
\end{align*}

\begin{itembox}[htbp]{ガウスの定理微分形}
\begin{align*}
\round{^2V}{x^2}+
\round{^2V}{y^2}+
\round{^2V}{z^2}=
-\frac{\rho}{\epsilon_0}
\end{align*}
\end{itembox}

\begin{align*}
\div\grad{V}=\Delta V=&
\left(\round{\vi}{x}+\round{\vj}{x}+\round{\vk}{x}\right)\cdot
\left(\round{V\vi}{x}+\round{V\vj}{x}+\round{V\vk}{x}\right)\\
=&\round{^2V}{x^2}+\round{^2V}{y^2}+\round{^2V}{z^2}\\
=&-\frac{\rho}{\epsilon_0} 
\end{align*}
これを\red{ポアソン方程式}という。
特に$\rho=0$のとき
\begin{align*}
\round{^2V}{x^2}+\round{^2V}{y^2}+\round{^2V}{z^2}=0
\end{align*}
である。これを\red{ラプラス方程式}という。

これは2階微分方程式なので、$V$を求めるには初期条件を2つ与える必要がある。

$V=ax+b$と表せるとき、
\begin{align*}
&\left\{
\begin{array}{l}
\round{^2V}{x^2}=0\\
V(x=0)=2\\
V(x=4)=6
\end{array}
\right.\\
&\left\{
\begin{array}{l}
b=2\\
4a+b=6\\
\end{array}
\right.
\Rightarrow a=1,b=2\\
&\therefore V=x+2\\
&E=-\round{V}{x}=-1
\end{align*}

\newenvironment{simul}[0]{\left\{ \begin{array}{l}}{\end{array} \right.}

\subsubsection*{円筒座標系}
$r,\phi,z$による座標系
\begin{align*}
\frac1r\round{}{r}\left(r\round{V}{r}\right)+
\frac1{r^2}\round{^2V}{\phi^2}+
\round{^2V}{\phi^2}
=-\frac\rho{\epsilon_0}
\end{align*}
\begin{align*}
\begin{simul}
x=r\cos\phi\\
y=r\sin\phi
\end{simul}
\end{align*}

\subsection{平行平面電極間の電位分布と電場}
$V=0$の平面電極と、それに平行な$V=V_a$の平面電極が距離$D$[m]離れたところにある。
2つの電極は$yz$平面上にある。

$x$成分のラプラス方程式($\rho=0$)
\begin{align*}
\round{^2V}{x^2}&=0\\
\int\round{^2V}{x^2}dx&=0\\
\Leftrightarrow\round{V}{x}&=c\;\;(c:定数)\\
\int\round{V}{x}dx&=\int cdx\\
\Leftrightarrow V&=cx+c'\;\;(c':定数)\\
V(x=0)=0,& V(x=D)=V_a\\
\begin{simul}
c'=0\\
cD+c'=V_a
\end{simul}
&\Rightarrow
\begin{simul}
c'=0\\
c=\frac{V_a}{D}
\end{simul}\\
\therefore V&=\frac{V_a}{D}x\\
\\
電場E_x&=-\round{V}{x}\;\;(E_y=E_z=0)\\
&=-\frac{V_a}{D}
\end{align*}

ここで、空間電荷密度$\rho=-kx^{-1/2}$とおく。
$x$軸方向のポアソン方程式は
\begin{align*}
\round{^2V}{x^2}=&\frac{k}{\epsilon_0}x^{-1/2}\\
\int \round{^2V}{x^2}dx=&\int \frac{k}{\epsilon_0}x^{-1/2}dx\\
\round Vx=&\frac{k}{\epsilon_0}\cdot 2x^{1/2}+c\;\;(c:定数)\\
\int \round Vx dx=&\frac{2k}{\epsilon_0}\int x^{1/2}dx+\int c'\;\;(c':定数)
\end{align*}
初期条件(境界条件)
\begin{align*}
V(x=0)=&0\Leftrightarrow c'=0\\
V(x=D)=&V_a\Leftrightarrow \frac{4k}{3\epsilon_0}D^{3/2}+cD+c'=V_a\\
\therefore cD=&V_a-\frac{4k}{3\epsilon_0}c^{1/2}\\
c=&\frac{V_a}{D}-\frac{4k}{3\epsilon_0}c^{1/2}
\end{align*}

\newcommand{\vF}[0]{\bm F}
\newcommand{\vr}[0]{\bm r}
\newcommand{\vv}[0]{\bm v}
\section{静電場中の電子の運動}
\subsection{電場による電子の加速}
空間中に電子がある。
ここに静電場$\vE=(E_x.E_y,E_z)$をかけると電子は電場の向きと反対向きに\red{クーロン力}$\vF=(F_x, F_y, F_z)$を受ける。
電子の電荷を$-e$[C],質量を$m$[kg]とすると、運動方程式より、
\begin{align*}
m\frac{d^2\vr}{dt^2}=\vF(=-e\vE)\;\;
(a\coloneqq\frac{d\vv}{dt},\vv\coloneqq\frac{d\vr}{dt})
\end{align*}
成分表示すると、
\begin{align*}
\begin{simul}
m\frac{d^2x}{dt^2}=-eE_x\\
m\frac{d^2y}{dt^2}=-eE_y\\
m\frac{d^2z}{dt^2}=-eE_z
\end{simul}
\end{align*}
電位を$V$として$\vE=-\nabla V$より、
\begin{align*}
\begin{simul}
m\frac{d^2x}{dt^2}=e\round{V}{x}\\
m\frac{d^2y}{dt^2}=e\round{V}{y}\\
m\frac{d^2z}{dt^2}=e\round{V}{z}
\end{simul}
\end{align*}

電子にも重力は働くが、クーロン力よりも十分小さいので無視する。

%%\\
%%電荷$Q_a,Q_b,Q_c$を三角形に、
%%距離をそれぞれ$r_{ab}, r_{bc}, r_{ca}$だけ離して配置する。
%%
%%このとき、
%%\begin{align*}
%%V_{ab}=
%%\end{align*}

\subsubsection{平行平面電極間の電子の運動}
例によってさっきから使っているコンデンサの、$V=0$のところに電子を1個おく。
電子の初速度は$v_0=0$である。
電子の運動方程式($\rho=0$)より、
\begin{align*}
m\frac{d^2x}{dt^2}=&e\frac{V_a}{D}\\
\Leftrightarrow \frac{d^2x}{dt^2}=&\frac{eV_a}{D}\\
これを積分して、\\
\int\frac{d^2x}{dt^2}dt=&\frac{eV_a}{D}\int dt\\
\frac xt =& \frac{eV_a}{mD}t+c\;\;(c:初速度(定数))
\end{align*}
この時、$v_0=0$より初速度$c$は0である。

もう一度積分して、
\begin{align*}
\int\ frac xt dt=& \frac{eV_a}{mD}t\int tdt\\
x=&\frac{eV_a}{mD}\frac{t^2}{2}+c'\;\;(c':初期位置(定数))\\
t=0\rightarrow x=&0\\
\therefore c'=&0\\
\therefore x=&\frac{eV_a}{2mD}t^2{\rm m}
\end{align*}

(別解)
\begin{align*}
v=&V_a+at\\
x=&V_0t+\frac12at^2\\
V_0=&0,a=\frac{eV_a}{mD}\\
\Rightarrow&
\begin{simul}
v=\frac{eV_a}{mD}t\\
x=\frac{eV_a}{2mD}t^2
\end{simul}
\end{align*}
これは等加速度運動である。

また、電子が-極から+極に到達するまでにかかる時間$\tau$は
\begin{align*}
D=\frac{eV_a}{2mD}\tau^2\Leftrightarrow &\tau^2=\frac{2mD^2}{eV_a}\\
\therefore \tau=&\sqrt{\frac{2m}{eV_a}}D
\end{align*}
$\tau$:\red{電子走行時間}($\rho=0,V_0=0$)
\end{document}

