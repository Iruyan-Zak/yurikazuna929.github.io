\documentclass[12pt]{ltjsarticle}
\usepackage{mymacro}
\usepackage{mathmode}
\usepackage{notemode}
\usepackage[ipaex]{luatexja-preset}
\begin{document}
\newcommand{\unit}[1]{{\rm \,\,\,[#1]}}
\subsection{電子ボルト及び電子の速度}
電子のEOM(一次元$x$軸方向)
\begin{align*}
m\frac{dv}{dt} &= e\frac{dV}{dx},\quad\sps{v=\frac{dx}{dt}}\\
\int mv\frac{dv}{dt}dt &= \int ev\frac{dV}{dx}dt\\
\int^v_{v_0} &= \int^{V_0}_0 edV\\
\Leftrightarrow \lps{\frac12mv^2}^v_{v_0} &= \lps{eV}^{V_0}_0\\
\frac12mv^2-\frac12mv_0^2 &= eV_0\\
(初速度v_0は0とする)& \\
\frac12mv^2 &= eV_0 \unit{J}
\end{align*}

電子が1 [V]の電位差の間を通った時に得られるエネルギーは$1.602\times10^{-19}$ [J]

これを\red{1 [eV(電子ボルト)]}と定義する。

電子の速度は
\begin{align*}
v=\sqrt\frac{2eV_0}{m} \unit{m/s}
\end{align*}

\clearpage
\section{静磁場中の電子の運動}
\subsection*{ベクトル積(外積)}
\newcommand{\va}{\bm A}
\newcommand{\vb}{\bm B}
\newcommand{\vc}{\bm C}
\newcommand{\vd}{\bm D}
\newcommand{\vi}{\bm i}
\newcommand{\vj}{\bm j}
\newcommand{\vk}{\bm k}
点$O$から\va,\vb が角度$\theta$をなして存在している。

\va と\vb によって作られる平行四辺形の面積$S$は、
\[ S=|\va||\vb|\sin\theta \]
このとき、平行四辺形に垂直に交わるベクトル\vc は、
\[ \vc = \va \times \vb = - \vb \times \va \]
となる。

よって、
\begin{align*}
\va\times\vb &= -\vb\times\va\\
(\va+\vb)\times\vd &= \va\times\vd + \vb\times\vd\\
\va=(A_x,A_y,A_z)&\quad \vb=(B_x,B_yB_z)\\
ここで、x,y,z方向の単位ベクトル&\vi,\vj,\vk を考えると、\\
\vi\times\vj=\vk,
\vj\times\vk=\vi&,
\vk\times\vi=\vj,
\vi\times\vi=\bm 0
\end{align*}
\vi,\vj,\vk は$Cyclic$である。

\begin{align*}
\va &= A_x\vi+A_y\vj+A_z\vk,
\vb = B_x\vi+B_y\vj+B_z\vk\\
\va\times\vb &=
\sps{A_x\vi+A_y\vj+A_z\vk}\times
\sps{B_x\vi+B_y\vj+B_z\vk}\\
&=A_xB_y\vi\times\vj
+A_xB_z\vi\times\vk
+A_yB_x\vj\times\vi
+A_yB_z\vj\times\vk
+A_zB_x\vk\times\vi
+A_zB_y\vk\times\vj\\
&=(A_yB_z-A_zB_y)\vi
+(A_zB_x-A_xB_z)\vj
+(A_xB_y-A_yB_x)\vk\\
&=(A_yB_z-A_zB_y,A_zB_x-A_xB_z,A_xB_y-A_yB_x)\\
&=\left|
\begin{array}{ccc}
\vi&\vj&\vk\\
A_x&A_y&A_z\\
B_x&B_y&B_z
\end{array}
\right|\\
&=\sum^3_{j,k=1}{\epsilon_{ijk}A_jB_k}
\end{align*}

\clearpage
\subsection{磁場による電子の加速}
\newcommand{\vf}{\bm F}
\newcommand{\vr}{\bm r}
\newcommand{\vv}{\bm v}
\bold{◎ローレンツ力}
磁場中で速度$v$ [m/s]で動く電子は力を受ける。
この力を\red{ローレンツ力}という。

\[ \vf = -e \vv\times\vb \quad(\vb:磁束密度) \]

電子のEOMは
\begin{align*}
\frac{d^2\vr}{dt^2} = -\frac em\vv\times\vb\\
成分表示:\vr=(x,y,z),\vv=(v_x,v_y,v_z),\vb=(B_xB_y,B_z)とおく。
\end{align*}
\begin{align*}
\begin{simul}
\frac{d^2x}{dt^2} = -\frac em\sps{v_yB_z-v_zB_y}\\
\frac{d^2y}{dt^2} = -\frac em\sps{v_xB_z-v_zB_x}\\
\frac{d^2z}{dt^2} = -\frac em\sps{v_xB_y-v_yB_x}
\end{simul}
\end{align*}
\end{document}

