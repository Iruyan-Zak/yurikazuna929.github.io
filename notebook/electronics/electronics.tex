\documentclass[12pt]{ltjsarticle}
\usepackage{mymacro}
\usepackage{mathmode}
\usepackage{notemode}
\newcommand{\ds}{\displaystyle}
\begin{document}
\section{光電子放出}
\subsection{光電子放出の一般の性質}
◎\red{光電子放出}
真空中の金属にエネルギー$\varepsilon$の光を当てると、\red{光電子}と呼ばれる電子が真空中に飛び出す。

光は電磁気で記述される・・・\red{電磁波}(波の性質を持つ)

波の振幅が大きいほど、波のエネルギーは大きい。

結局電子が金属中から真空中に飛び出すためには金属にエネルギーを与えて、そのエネルギーを障壁のポテンシャルエネルギー以上にしないといけない。

与えないといけないエネルギーは仕事関数$\phi$なので、電子を真空中に放出したいなら、
\begin{align*}
\varepsilon \ge \phi
\end{align*}
となるくらいの強い光を与えないといけない。


\begin{itembox}[l]{光電子放出に関するいろんな実験}
\begin{enumerate}
\item 電子が出てくる条件

光は振幅と周波数(振動数)$\nu$で特徴づけられる。
\begin{enumerate}
\item $\nu\ge\nu_0$のとき($\nu_0$は\red{限界周波数})

光電子放出が起きる。

\item $\nu<\nu_0$のとき

光電子放出が起きない。
しかも振幅(光のエネルギー)とは無関係(速い波じゃないとダメ)
\end{enumerate}

\item 出てくる電子の数

$\nu\ge\nu_0$で光のエネルギーを大きくすると、
\underline{電流は大きくなる。}(出てくる電子の数が増える)
\begin{itembox}[l]{アインシュタインの光量子仮説}

光は粒子としての性質を持ち、1個の光の粒子\red{光子}が周波数$\nu$を持つとき、
エネルギー$\varepsilon$は$\varepsilon=h\nu$となる。($h$は\red{プランク定数})
\end{itembox}
光は波だけでなく、粒子の性質をも併せ持つ(\red{光の二重性})

\end{enumerate}
\end{itembox}


光の周波数$\nu$と仕事関数$\phi$との関係
\begin{align*}
&h\nu\ge\phi\\
\Leftrightarrow&\nu\ge\frac\phi h=\red{\nu_0}
\end{align*}

この時の波長は$v=f\lambda$より、
\begin{align*}
\lambda_0=\frac c{\nu_0}=\frac{ch}\phi
\end{align*}
$c$ : 光速\\
$\lambda_0$ : \red{限界波長}


フェルミ準位にある電子が光電子放出するとき、その速度は最高速度$v_m$となる。
(他の金属中の電子よりも速い)

エネルギー保存則より、
\begin{align*}
\frac12mv_m^2&=E_F+\varepsilon-(E_F+\phi)\\
&=\varepsilon-\phi\\
&=h\nu-h\nu_0\\
&=h(\nu-\nu_0)\\
\therefore v_m&=\sqrt{\frac{2h}m (\nu-\nu_0)}
\end{align*}

他の光の周波数に依存し、振幅には依存しない現象

例)日焼け
\begin{itemize}
\item 電気ストーブ(赤外線)→日焼けしない(周波数が小さい)
\item 日光(紫外線)→日焼けする(周波数が大きい)
\end{itemize}

\clearpage
\section{電位分布と電場}
\subsection{電位と電場の関係}
電場$\vec E(\bm E)$電位$V(\bm r)$の時、
\begin{align*}
\bm E&=-\nabla V(\bm r)\\
&=\grad V(\bm r)
\end{align*}
grad:\red{勾配}

\newcommand{\vi}[0]{\bm i}
\newcommand{\vj}[0]{\bm j}
\newcommand{\vk}[0]{\bm k}
\newcommand{\vE}[0]{\bm E}

\begin{align*}
\grad \equiv \nabla &\equiv
(\round{}x,\round{}y,\round{}z)\\
&=(\partial_x,\partial_y,\partial_z)\\
&=\round{}x\vi+\round{}y\vj+\round{}z\vk\\
\end{align*}
$\vi,\vj,\vk$:単位ベクトル
\begin{align*}
|\vi|=|\vj|=|\vk|=1\\
\vi\cdot\vj=\vj\cdot\vk=\vk\cdot\vi=0\\
\vi=(1,0,0),\vj=(0,1,0),\vk=(0,0,1)
\end{align*}
$\bm E=(E_x,E_y,E_z)$を$x,y,z$の成分表示すると、
\begin{align*}
E_x=-\round Vx,E_y=-\round Vy,E_z=-\round Vz
\end{align*}
$V$:スカラー $\vE$:ベクトル

\subsection{電位分布を求めるための基礎方程式}
電荷が連続的に分布する球体(\red{空間電荷})を考える。

この時、単位体積あたりの電荷量\rho[C/m$^3$]を\red{空間電荷密度}という。

半径rの球体の全電荷量Qは
\begin{align*}
Q=\frac43\pi r^3\rho
\end{align*}
となる。

\subsection*{ガウスの法則}
正の電荷からは電気力線が湧き出し、負の電荷へは電気力線が吸い込まれる。
電気力線の束を\red{電束}といい、
1[C]の電荷からは1本の電束が出ているものと定義する。

微小体積を貫く電束を考える。
ここでは簡単にするために$x$方向に貫く電束のみを考える。

$x$方向の電場の変化量は
\begin{align*}
E_x\rightarrow E_x+\round{E_x}{x}dx
\end{align*}
これは微小体積での$x$方向の電場の変化量の変化を表す。

\begin{align*}
(電束の変化)&=\epsilon_0\round{E_x}{x}dxdydz\\
\\
\epsilon_0\round{E_x}{x}dx&:電束密度\\
dydz&:貫く面積\\
\epsilon_0&:真空の誘電率 [{\rm C^2/(N\cdot m^2)}]\\
\\
&=\rho dxdydz(全電荷)
\end{align*}

$y,z$方向も考えると、
\begin{align*}
&\epsilon_0\left[
\epsilon_0\round{E_x}{x}dxdydz+
\epsilon_0\round{E_y}{y}dxdydz+
\epsilon_0\round{E_z}{z}dxdydz
\right]\\
=&\epsilon_0\left[\round{E_x}{x}+\round{E_y}{y}+\round{E_z}{z}\right]dxdydz\\
=&\rho dxdydz
\end{align*}
よって、ガウスの法則

\begin{align*}
\div{\vE}=&\frac{\rho}{\epsilon_0}(=\nabla\cdot\vE)\\
\div=&(\round{}{x}+\round{}{y}+\round{}{z})
=\round{\vi}{x}+\round{\vj}{x}+\round{\vk}{x}\\
\vE=&-\grad{V}
\end{align*}

\begin{itembox}[htbp]{ガウスの定理微分形}
\begin{align*}
\round{^2V}{x^2}+
\round{^2V}{y^2}+
\round{^2V}{z^2}=
-\frac{\rho}{\epsilon_0}
\end{align*}
\end{itembox}

\begin{align*}
\div\grad{V}=\Delta V=&
\left(\round{\vi}{x}+\round{\vj}{x}+\round{\vk}{x}\right)\cdot
\left(\round{V\vi}{x}+\round{V\vj}{x}+\round{V\vk}{x}\right)\\
=&\round{^2V}{x^2}+\round{^2V}{y^2}+\round{^2V}{z^2}\\
=&-\frac{\rho}{\epsilon_0} 
\end{align*}
これを\red{ポアソン方程式}という。
特に$\rho=0$のとき
\begin{align*}
\round{^2V}{x^2}+\round{^2V}{y^2}+\round{^2V}{z^2}=0
\end{align*}
である。これを\red{ラプラス方程式}という。

これは2階微分方程式なので、$V$を求めるには初期条件を2つ与える必要がある。

$V=ax+b$と表せるとき、
\begin{align*}
&\left\{
\begin{array}{l}
\round{^2V}{x^2}=0\\
V(x=0)=2\\
V(x=4)=6
\end{array}
\right.\\
&\left\{
\begin{array}{l}
b=2\\
4a+b=6\\
\end{array}
\right.
\Rightarrow a=1,b=2\\
&\therefore V=x+2\\
&E=-\round{V}{x}=-1
\end{align*}

%\newenvironment{simul}[0]{\left\{ \begin{array}{l}}{\end{array} \right.}

\subsubsection*{円筒座標系}
$r,\phi,z$による座標系
\begin{align*}
\frac1r\round{}{r}\left(r\round{V}{r}\right)+
\frac1{r^2}\round{^2V}{\phi^2}+
\round{^2V}{\phi^2}
=-\frac\rho{\epsilon_0}
\end{align*}
\begin{align*}
\begin{simul}
x=r\cos\phi\\
y=r\sin\phi
\end{simul}
\end{align*}

\subsection{平行平面電極間の電位分布と電場}
$V=0$の平面電極と、それに平行な$V=V_a$の平面電極が距離$D$[m]離れたところにある。
2つの電極は$yz$平面上にある。

$x$成分のラプラス方程式($\rho=0$)
\begin{align*}
\round{^2V}{x^2}&=0\\
\int\round{^2V}{x^2}dx&=0\\
\Leftrightarrow\round{V}{x}&=c\;\;(c:定数)\\
\int\round{V}{x}dx&=\int cdx\\
\Leftrightarrow V&=cx+c'\;\;(c':定数)\\
V(x=0)=0,& V(x=D)=V_a\\
\begin{simul}
c'=0\\
cD+c'=V_a
\end{simul}
&\Rightarrow
\begin{simul}
c'=0\\
c=\frac{V_a}{D}
\end{simul}\\
\therefore V&=\frac{V_a}{D}x\\
\\
電場E_x&=-\round{V}{x}\;\;(E_y=E_z=0)\\
&=-\frac{V_a}{D}
\end{align*}

ここで、空間電荷密度$\rho=-kx^{-1/2}$とおく。
$x$軸方向のポアソン方程式は
\begin{align*}
\round{^2V}{x^2}=&\frac{k}{\epsilon_0}x^{-1/2}\\
\int \round{^2V}{x^2}dx=&\int \frac{k}{\epsilon_0}x^{-1/2}dx\\
\round Vx=&\frac{k}{\epsilon_0}\cdot 2x^{1/2}+c\;\;(c:定数)\\
\int \round Vx dx=&\frac{2k}{\epsilon_0}\int x^{1/2}dx+\int c'\;\;(c':定数)
\end{align*}
初期条件(境界条件)
\begin{align*}
V(x=0)=&0\Leftrightarrow c'=0\\
V(x=D)=&V_a\Leftrightarrow \frac{4k}{3\epsilon_0}D^{3/2}+cD+c'=V_a\\
\therefore cD=&V_a-\frac{4k}{3\epsilon_0}c^{1/2}\\
c=&\frac{V_a}{D}-\frac{4k}{3\epsilon_0}c^{1/2}
\end{align*}


\clearpage
\newcommand{\vF}[0]{\bm F}
\newcommand{\vr}[0]{\bm r}
\newcommand{\vv}[0]{\bm v}
\section{静電場中の電子の運動}
\subsection{電場による電子の加速}
空間中に電子がある。
ここに静電場$\vE=(E_x.E_y,E_z)$をかけると電子は電場の向きと反対向きに\red{クーロン力}$\vF=(F_x, F_y, F_z)$を受ける。
電子の電荷を$-e$[C],質量を$m$[kg]とすると、運動方程式より、
\begin{align*}
m\frac{d^2\vr}{dt^2}=\vF(=-e\vE)\;\;
(a\coloneqq\frac{d\vv}{dt},\vv\coloneqq\frac{d\vr}{dt})
\end{align*}
成分表示すると、
\begin{align*}
\begin{simul}
m\frac{d^2x}{dt^2}=-eE_x\\
m\frac{d^2y}{dt^2}=-eE_y\\
m\frac{d^2z}{dt^2}=-eE_z
\end{simul}
\end{align*}
電位を$V$として$\vE=-\nabla V$より、
\begin{align*}
\begin{simul}
m\frac{d^2x}{dt^2}=e\round{V}{x}\\
m\frac{d^2y}{dt^2}=e\round{V}{y}\\
m\frac{d^2z}{dt^2}=e\round{V}{z}
\end{simul}
\end{align*}

電子にも重力は働くが、クーロン力よりも十分小さいので無視する。

\subsubsection{平行平面電極間の電子の運動}
例によってさっきから使っているコンデンサの、$V=0$のところに電子を1個おく。
電子の初速度は$v_0=0$である。
電子の運動方程式($\rho=0$)より、
\begin{align*}
m\frac{d^2x}{dt^2}=&e\frac{V_a}{D}\\
\Leftrightarrow \frac{d^2x}{dt^2}=&\frac{eV_a}{D}\\
これを積分して、\\
\int\frac{d^2x}{dt^2}dt=&\frac{eV_a}{D}\int dt\\
\frac xt =& \frac{eV_a}{mD}t+c\;\;(c:初速度(定数))
\end{align*}
この時、$v_0=0$より初速度$c$は0である。

もう一度積分して、
\begin{align*}
\int\ frac xt dt=& \frac{eV_a}{mD}t\int tdt\\
x=&\frac{eV_a}{mD}\frac{t^2}{2}+c'\;\;(c':初期位置(定数))\\
t=0\rightarrow x=&0\\
\therefore c'=&0\\
\therefore x=&\frac{eV_a}{2mD}t^2{\rm m}
\end{align*}

(別解)
\begin{align*}
v=&V_a+at\\
x=&V_0t+\frac12at^2\\
V_0=&0,a=\frac{eV_a}{mD}\\
\Rightarrow&
\begin{simul}
v=\frac{eV_a}{mD}t\\
x=\frac{eV_a}{2mD}t^2
\end{simul}
\end{align*}
これは等加速度運動である。

また、電子が-極から+極に到達するまでにかかる時間$\tau$は
\begin{align*}
D=\frac{eV_a}{2mD}\tau^2\Leftrightarrow &\tau^2=\frac{2mD^2}{eV_a}\\
\therefore \tau=&\sqrt{\frac{2m}{eV_a}}D
\end{align*}
$\tau$:\red{電子走行時間}($\rho=0,V_0=0$)

\newcommand{\unit}[1]{{\rm \,\,\,[#1]}}
\subsection{電子ボルト及び電子の速度}
電子のEOM(一次元$x$軸方向)
\begin{align*}
m\frac{dv}{dt} &= e\frac{dV}{dx},\quad\sps{v=\frac{dx}{dt}}\\
\int mv\frac{dv}{dt}dt &= \int ev\frac{dV}{dx}dt\\
\int^v_{v_0} &= \int^{V_0}_0 edV\\
\Leftrightarrow \lps{\frac12mv^2}^v_{v_0} &= \lps{eV}^{V_0}_0\\
\frac12mv^2-\frac12mv_0^2 &= eV_0\\
(初速度v_0は0とする)& \\
\frac12mv^2 &= eV_0 \unit{J}
\end{align*}

電子が1 [V]の電位差の間を通った時に得られるエネルギーは$1.602\times10^{-19}$ [J]

これを\red{1 [eV(電子ボルト)]}と定義する。

電子の速度は
\begin{align*}
v=\sqrt\frac{2eV_0}{m} \unit{m/s}
\end{align*}

\clearpage
\section{静磁場中の電子の運動}
\subsection*{ベクトル積(外積)}
\newcommand{\va}{\bm A}
\newcommand{\vb}{\bm B}
\newcommand{\vc}{\bm C}
\newcommand{\vd}{\bm D}
%\newcommand{\vi}{\bm i}
%\newcommand{\vj}{\bm j}
%\newcommand{\vk}{\bm k}
点$O$から\va,\vb が角度$\theta$をなして存在している。

\va と\vb によって作られる平行四辺形の面積$S$は、
\[ S=|\va||\vb|\sin\theta \]
このとき、平行四辺形に垂直に交わるベクトル\vc は、
\[ \vc = \va \times \vb = - \vb \times \va \]
となる。

よって、
\begin{align*}
\va\times\vb &= -\vb\times\va\\
(\va+\vb)\times\vd &= \va\times\vd + \vb\times\vd\\
\va=(A_x,A_y,A_z)&\quad \vb=(B_x,B_yB_z)\\
ここで、x,y,z方向の単位ベクトル&\vi,\vj,\vk を考えると、\\
\vi\times\vj=\vk,
\vj\times\vk=\vi&,
\vk\times\vi=\vj,
\vi\times\vi=\bm 0
\end{align*}
\vi,\vj,\vk は$Cyclic$である。

\begin{align*}
\va &= A_x\vi+A_y\vj+A_z\vk,
\vb = B_x\vi+B_y\vj+B_z\vk\\
\va\times\vb &=
\sps{A_x\vi+A_y\vj+A_z\vk}\times
\sps{B_x\vi+B_y\vj+B_z\vk}\\
&=A_xB_y\vi\times\vj
+A_xB_z\vi\times\vk
+A_yB_x\vj\times\vi
+A_yB_z\vj\times\vk
+A_zB_x\vk\times\vi
+A_zB_y\vk\times\vj\\
&=(A_yB_z-A_zB_y)\vi
+(A_zB_x-A_xB_z)\vj
+(A_xB_y-A_yB_x)\vk\\
&=(A_yB_z-A_zB_y,A_zB_x-A_xB_z,A_xB_y-A_yB_x)\\
&=\left|
\begin{array}{ccc}
\vi&\vj&\vk\\
A_x&A_y&A_z\\
B_x&B_y&B_z
\end{array}
\right|\\
&=\sum^3_{j,k=1}{\epsilon_{ijk}A_jB_k}
\end{align*}

\clearpage
\subsection{磁場による電子の加速}
\newcommand{\vf}{\bm F}
%\newcommand{\vr}{\bm r}
%\newcommand{\vv}{\bm v}
\bold{◎ローレンツ力}
磁場中で速度$v$ [m/s]で動く電子は力を受ける。
この力を\red{ローレンツ力}という。

\[ \vf = -e \vv\times\vb \quad(\vb:磁束密度) \]

電子のEOMは
\begin{gather*}
\frac{d^2\vr}{dt^2} = -\frac em\vv\times\vb\\
成分表示:\vr=(x,y,z),\vv=(v_x,v_y,v_z),\vb=(B_xB_y,B_z)とおく。
\end{gather*}
\begin{align*}
\begin{simul}
\ds\frac{d^2x}{dt^2} = -\frac em\sps{v_yB_z-v_zB_y}\\
\ds\frac{d^2y}{dt^2} = -\frac em\sps{v_xB_z-v_zB_x}\\
\ds\frac{d^2z}{dt^2} = -\frac em\sps{v_xB_y-v_yB_x}
\end{simul}
\end{align*}


\newcommand{\vB}{{\bm B}}
%\setlength\abovedisplayskip{16pt plus 3pt minus 7pt}
%\setlength\belowdisplayskip{16pt plus 3pt minus 7pt}
%\makeatletter
%\g@addto@macro\normalsize{%
%  \setlength\abovedisplayskip{40pt}
%    \setlength\belowdisplayskip{40pt}
%      \setlength\abovedisplayshortskip{40pt}
%        \setlength\belowdisplayshortskip{40pt}
%}
%\makeatother
\subsubsection{一様な磁場中の電子の運動}
\bold{◎ローレンツ力}
\[
    \vf=-e\vv\times\vB
\]
電子のEOMは、
\[\round{^2\vr}{t^2}=-e\vv\times\vB, \vv=\frac{d\vr}{dt}\]

\begin{align*}
\vr=(x,y,z),\quad \vB=(B_x,B_y,B_z)\\
\begin{simul}
\ds\frac{d^2x}{dt^2}=-\frac em\sps{\frac{dy}{dt}B_z-\frac{dz}{dt}B_y}\\
\ds\frac{d^2y}{dt^2}=-\frac em\sps{\frac{dz}{dt}B_x-\frac{dx}{dt}B_z}\\
\ds\frac{d^2z}{dt^2}=-\frac em\sps{\frac{dx}{dt}B_y-\frac{dy}{dt}B_x}\\
\end{simul}
\end{align*}

$Z$軸方向の磁束密度を考える。($\vB=(0,0,B_z)$)

\begin{align*}
\begin{simul}
\ds\frac{d^2x}{dt^2}=-\frac em\frac{dy}{dt}B_z\\
\ds\frac{d^2x}{dt^2}=-\frac em\frac{dx}{dt}B_z\\
\ds\frac{d^z}{dt^2}=0
\end{simul}
\end{align*}

これらを$t$で積分する。

\begin{align*}
\frac{dx}{dt}&=-\frac emyB_z+C_2'\\
\frac{dx}{dt}&=-\frac emxB_z+C_1'\\
\frac{dz}{dt}&=C_3\\
&(C_1',C_2',C_3'は積分定数)\\
C_2'&=-\frac emB_zC_2,C_1'=\frac emB_zC_1\quad(C_1,C_2は定数)とすると
\end{align*}

\begin{align}
\label{eq1}\frac{dx}{dt}&=-\frac emB_z(y+C_2)=v_x\\
\label{eq2}\frac{dx}{dt}&=-\frac emB_z(x+C_1)=v_y\\
\label{eq3}\frac{dz}{dt}&=C_3=v_z
\end{align}

\begin{align*}
v_z&=v\cos\theta\quad(v=|\vv|)\\
\sqrt{v_x^2+v_y^2}&=v\sin\theta
\Leftrightarrow v_x^2+v_y^2=v^2\sin^2\theta\\
(\ref{eq1}),(\ref{eq2})を&代入すると、\\
&\frac{e^2}{m^2}B_z^2(y+C_2)^2
\frac{e^2}{m^2}B_z^2(x+C_1)^2
=v^2\sin^2\theta\\
&(x+C_1)^2+(y+C_2)^2=\sps{\frac{mv\sin\theta}{eB_z}}^2\\
&円の方程式:半径r=\frac{mv\sin\theta}{eB_z}\\
周期Tは&T=\frac{a\pi r}{v\sin\theta}=\frac{2\pi m}{eB_z}\\
角速度\omega_c&=\frac{2\pi}{T}=\frac{eB_z}{m}
:{\rm \red{サイクロトロン角周波数}}
\end{align*}

$Z$軸が円の中心を通るとき、$C_1=0,C_2=0$となる。その時の解は、

\begin{align*}
\begin{simul}
x=r\cos(\omega_ct+\epsilon)\\
x=r\sin(\omega_ct+\epsilon)\\
z=vt\cos\theta
\end{simul}\\
\epsilon はt=0の電子の位置で決まる。
\end{align*}

これは$z$軸を中心に螺旋運動する。

\clearpage
\subsubsection{静電磁場中の電子の運動}
\bold{◎電場と磁場が直交しているとき}
距離$D$,電位差$v_a$のコンデンサを考える。電子のEOMは、
\begin{align*}
\vE&=(E,0,0)\\
\vB&=(0,0,B)
\end{align*}

教科書の(3.24),(3.25),(3.37)式を組み合わせると、
\begin{align*}
m\frac{d^2\vr}{dt^2}&=-e\vE-e\vv\times\vB\\
E&=-\round{V_x}{x}=-\frac{V_a}{D}
\end{align*}

成分表示すると、
\newcommand{\blue}[1]{\textcolor{blue}{#1}}
\newcommand{\nl}{\\[10pt]}
\begin{align*}
\begin{simul}
\ds\frac{d^2x}{dt^2}=\red{\frac em\frac{V_a}{D}}
-\blue{\frac emB\frac{dy}{dt}}=\frac{dv_z}{dt}\nl
\ds\frac{d^2y}{dt^2}=\blue{\frac emB\frac{dy}{dt}}=\frac{dv_y}{dt}\nl
\ds\frac{d^2z}{dt^2}=0,\frac{dv_z}{dt}=0\quad(z軸方向は等速直線運動)
\end{simul}
\end{align*}
(\red{赤字:クーロン力},\blue{青字:ローレンツ力})

ここで$\ds A=\red{\frac{eV_a}{mD}},\omega_c=\blue{\frac{eB}{m}}$とおく。

\begin{align*}
\frac{dv_x}{dt}=A-\omega_cv_y\\
tで微分して、\frac{d^2V_x}{dt^2}=-\omega_c\frac{dv_y}{dt}\\
\frac{dv_y}{dt}=\omega_cv_x\\
\Rightarrow \frac{d^v_x}{dt^2}=-\omega_c^2v_x
\end{align*}

これを$v_x=\cdots$の式に直し、$v_x$についての微分方程式を解く。
\begin{enumerate}
\item 上の式を2階積分して$v_x=\cdots$の式にする
\item 一般解を与える(微分方程式を満たすように解を決める)
\end{enumerate}

\begin{align*}
v_x&=C_1\sin{\omega_ct}+C_2\cos{\omega_ct},\quad (C_1,C_2:積分定数)\\
\frac{dv_x}{dt}&=\omega_c(C_1\cos{\omega_ct}-C_2\sin{\omega_ct})\\
\frac{d^2v_x}{dt^2}&=
-\omega_c^2(C_1\sin{\omega_ct}+C_2\cos{\omega_ct})\\
&=-\omega_c^2v_x
\end{align*}

\end{document}

