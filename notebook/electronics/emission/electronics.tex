\documentclass[12pt]{jsarticle}
\usepackage[top=15truemm,bottom=15truemm,left=10truemm,right=10truemm]{geometry}
\usepackage[dvipdfmx]{graphicx}
% \def\pgfsysdriver{pgfsys-pdftex.def}%(graphicxパッケージを使用しない場合)
\usepackage{amsmath}
\usepackage{txfonts}
\usepackage{mathptmx}
\usepackage{indentfirst}
\usepackage{color}
\usepackage{ascmac}
%\usepackage[svgnames]{xcolor}%tikzパッケージよりも前に読み込みます。
%\usepackage{tikz}
\begin{document}
\renewcommand{\rmdefault}{ptm}
\renewcommand{\sfdefault}{phv}
\renewcommand{\ttdefault}{pcr}
\newcommand{\red}[1]{\textcolor{red}{#1}}
\normalfont
\renewcommand{\labelenumi}{(\arabic{enumi})}


\subsection{ダッシュマンリチャードソンの式}
熱電子放出より、金属の表面から放出される熱電子の電流密度を表す式。

仕事関数$\phi$とフェルミ準位$E_F$の関係
自由電子が真空に飛び出す条件は
\begin{align*}
\frac 1 2mv^2_x=\frac{p_x^2}{2m}&\ge E_F+\phi,\;(p_x=mv_x)\\
p_x&\ge \sqrt{2m(E_F+\phi)}=p_x
\end{align*}
である。

熱電子の電流密度Jは
\begin{align*}
J=e\frac NVv_x
\end{align*}
$\frac NVv_x$・・・単位体積あたりの電子の数
$e$,電荷,
$V$,金属の体積,
$N$,電子の数

\begin{align*}
N&=\int^\infty_0n(E)dE\;\;(p_x\ge p_x')\\
n(E)&=n(E)=4\pi V(\frac{2m}{h^2})^{3/2}\sqrt{E}
\end{align*}

フェルミ分布関数($E-E_F \gg kT$の時)
\begin{align*}
F(E)&=\frac 1{1+\exp{\left[\frac{E-E_F}{kT}\right]}}\\
F(E)&=\frac 1{\exp{\left[\frac{E-E_F}{kT}\right]}}\\
F(E)&=\exp{\left[-\frac{E-E_F}{kT}\right]}
\end{align*}

ここで、エネルギー$E$は、
\begin{align*}
E&=\frac 1{2m}(p_x^2+p_y^2+p_z^2)\\
&=\frac 1{2m}p^2,\;\;(p^2=_x^2+p_y^2+p_z^2)\\
dE&=\frac 1{2m}\dot 2dp\\
\therefore J&=\frac eVv_xN\\
&=\frac eV\int v_x4\pi V(\frac{2m}{h^2})^{3/2}\sqrt{E}F(E)\\
&=e\frac{4\pi(\frac{2m}{h^2})^{3/2}}{m^2\sqrt{2m}}\int p_xF(p)p^2dp,\;\;(p_x\ge p_x')\\
\end{align*}
ここで$p^2dp=\frac1{4\pi}dp_xdp_ydp_z$とすると
\begin{align*}
J=
e\frac{4\pi(\frac{2m}{h^2})^{3/2}}{m^2\sqrt{2m}}
\frac1{4\pi}
\int^\infty_{p_x}dp_x
\int^\infty_{-\infty}dp_y
\int^\infty_{-\infty}dp_z
\times p_x
\exp\left[\frac{E_F}{kT}\right]
\exp\left[-\frac{p_x}{2mkT}\right]
\exp\left[-\frac{p_y}{2mkT}\right]
\exp\left[-\frac{p_z}{2mkT}\right]
\end{align*}


\clearpage
\section{光電子放出}
\subsection{光電子放出の一般の性質}
◎\red{光電子放出}
真空中の金属にエネルギー$\varepsilon$の光を当てると、\red{光電子}と呼ばれる電子が真空中に飛び出す。

光は電磁気で記述される・・・\red{電磁波}(波の性質を持つ)

波の振幅が大きいほど、波のエネルギーは大きい。

結局電子が金属中から真空中に飛び出すためには金属にエネルギーを与えて、そのエネルギーを障壁のポテンシャルエネルギー以上にしないといけない。

与えないといけないエネルギーは仕事関数$\phi$なので、電子を真空中に放出したいなら、
\begin{align*}
\varepsilon \ge \phi
\end{align*}
となるくらいの強い光を与えないといけない。


\begin{itembox}[l]{光電子放出に関するいろんな実験}
\begin{enumerate}
\item 電子が出てくる条件

光は振幅と周波数(振動数)$\nu$で特徴づけられる。
\begin{enumerate}
\item $\nu\ge\nu_0$のとき($\nu_0$は\red{限界周波数})

光電子放出が起きる。

\item $\nu<\nu_0$のとき

光電子放出が起きない。
しかも振幅(光のエネルギー)とは無関係(速い波じゃないとダメ)
\end{enumerate}

\item 出てくる電子の数

$\nu\ge\nu_0$で光のエネルギーを大きくすると、
\underline{電流は大きくなる。}(出てくる電子の数が増える)
\begin{itembox}[l]{アインシュタインの光量子仮説}

光は粒子としての性質を持ち、1個の光の粒子\red{光子}が周波数$\nu$を持つとき、
エネルギー$\varepsilon$は$\varepsilon=h\nu$となる。($h$は\red{プランク定数})
\end{itembox}
光は波だけでなく、粒子の性質をも併せ持つ(\red{光の二重性})

\end{enumerate}
\end{itembox}


光の周波数$\nu$と仕事関数$\phi$との関係
\begin{align*}
&h\nu\ge\phi\\
\Leftrightarrow&\nu\ge\frac\phi h=\red{\nu_0}
\end{align*}

この時の波長は$v=f\lambda$より、
\begin{align*}
\lambda_0=\frac c{\nu_0}=\frac{ch}\phi
\end{align*}
$c$ : 光速\\
$\lambda_0$ : \red{限界波長}


フェルミ準位にある電子が光電子放出するとき、その速度は最高速度$v_m$となる。
(他の金属中の電子よりも速い)

エネルギー保存則より、
\begin{align*}
\frac12mv_m^2&=E_F+\varepsilon-(E_F+\phi)\\
&=\varepsilon-\phi\\
&=h\nu-h\nu_0\\
&=h(\nu-\nu_0)\\
\therefore v_m&=\sqrt{\frac{2h}m (\nu-\nu_0)}
\end{align*}

他の光の周波数に依存し、振幅には依存しない現象

例)日焼け
\begin{itemize}
\item 電気ストーブ(赤外線)→日焼けしない(周波数が小さい)
\item 日光(紫外線)→日焼けする(周波数が大きい)
\end{itemize}

\end{document}
