\documentclass[12pt,titlepage]{ltjsarticle}
\usepackage{notemode}
\usepackage{listings}
\usepackage[setpagesize=false,pdfencoding=auto]{hyperref}

\lstset{basicstyle=\ttfamily\footnotesize,frame=single}
\newcommand{\fig}[2]{\begin{figure}[htbp]%
\centering\includegraphics[width=.5\textwidth]{fig/fig#1.png}%
\caption{#2}\end{figure}}


\begin{document}
\title{Processingの遊び方}
\author{Yuri Kazuna}
\maketitle

\section{一歩目}
\subsection{Processingとは}
Processingはプログラミング言語の一つです。
もうちょっと詳しく言うと、
Javaの機能を制限してお絵かきに特化させたものです。
書き方はC言語によく似てるので、馴染みやすいかと思います。
C言語を覚えていなくてもそれはそれで大丈夫です。

\subsection{Processingを拾う}
\href{https://processing.org/download/?processing}{https://processing.org/download/?processing}

\fig{1-1}{Processing置き場}

ここに、Processingがあります。
使っているOSにあわせてダウンロードしましょう。

ちなみにこのProcessingのバージョンは3.0です。
この8月にリリースされたので、Web上の情報と実際の動きが合わない
(情報が古い)ことがあります。
変だなあと思ったらゆーりに相談してください。

\subsection{Processingを動かす}
ダウンロードした書庫(zipとかtar.gzとか)を解凍すると、
processing(.exe)というファイルがあると思います。
それを実行しましょう。(インストール不要なんです)

\fig{1-2}{ポップアップウィンドウは消して}

そうするとポップアップウィンドウに「Get Started」みたいなのがあるので、
それをクリックしてポップアップウィンドウを消します。

\fig{1-3}{エディタ画面と向き合う}

エディタ画面がありますね?
それの左上にある三角形を押しましょう。

すると小さなウィンドウが出てきましたね?

\fig{1-4}{実行画面}

おめでとうございます!あなたの書いたコードが動きました!
(えっ、まだ何も書いてないって?)

\section{二歩目}
C言語では黒い画面を相手にプログラムを書いていましたが、
Processingでは今出てきた実行画面のウィンドウを相手にします。

二歩目では、ウィンドウにお絵かきをする方法を紹介していきます。

\subsection{文法}
関数の呼び出し(実行)は、

\verb#関数名(引数をいくつか);#

というふうにします。
引数の数が3だと、カッコの中は(引数1, 引数2, 引数3)となります。

実行の仕方は次章で説明しますが、この章で扱う内容を試したいなら、
Processingのエディタ画面に直接、

\verb#ellipse(30,30,40,40);#

などと書けばよいです。
(この時、関数名の部分がellipseに変わって、
いくつかの引数がが4つの数字に変わったと考える)
main関数などは必要ありません。

書いたらエディタ画面左上の三角形を押して実行です。

\subsection{図形}

\fig{2-1}{よくあるブロック崩し}

上のよくある感じのブロック崩しを考えてみましょう。
見えるものはなんですか?

「ボールの座標」?
それは見えてないですよね。

「ボール」?
ボールなんて描いた記憶が無いですね。

円ですね。円です。

あとは、四角、線、文字。

これだけあれば結構なものが表現できます。
画像については後述します。

\subsubsection{円を描く:ellipse(float,float,float,float)}
色指定:stroke, fill

円を描くにはellipse関数を使います。

\verb#ellipse(中心のX座標, 中心のY座標, 幅, 高さ);#

\subsubsection{長方形を描く:rect(float,float,float,float)}
色指定:stroke, fill

長方形を描くにはrect関数を使います。

\verb#rect(左上のX座標, 左上のY座標, 幅, 高さ);#

\subsubsection{直線を描く:ellipse(float,float,float,float)}
色指定:stroke

直線を描くにはline関数を使います。

\verb#line(始点のX座標, 始点のY座標, 終点のX座標, 終点のY座標);#

\subsubsection{文字を描く:text(string,float,float)}
色指定:fill

文字を画面に描くにはtext関数を使います。

\verb#text("書きたい文字列", 左上のX座標, 左上のY座標);#

書きたい文字列は半角ダブルクオーテーション
(", Shiftを押しながら2を押すと出ることが多い)
でくくらないと表示できないと思います。

\subsection{色}
ブロック崩しの例では黒と白しかありませんが、
なにかを作るときには黒と白以外の色も必要になりますね。

以下に、色を扱う関数を紹介します。
\subsubsection{画面を塗りつぶす:background(float)/background(float,float,float)}
画面を1色で塗りつぶしたいときにはbackground関数を使います。

ところで、上には関数の引数のとり方が2種類あるように書いてますが、
実際そうです。
以下に、使い方を示します。

\verb#background(明度);#


または

\verb#background(赤の強さ(R), 緑の強さ(G), 青の強さ(B));#

下の例のようにRGB値で背景色を指定する場合、
それぞれの値を同じ値にすると、
その色を明度とした上の例での塗りつぶしと同じ色で塗りつぶされます。

???と思ったら、やってみるのが早いです。

また、各値の最小は0、最大は255です。
それ以下・以上は無視されます。

\subsubsection{線の色を変える:stroke(float)/stroke(float,float,float)}
前項で示した図形のうち、色指定にstrokeを含むものの色を変えます。
先ほどと同様に色指定の方法は2種類あります。

\verb#stroke(明度);#


または

\verb#stroke(R,G,B);#

この関数は\bold{stroke関数以降の図形関数の色だけ}を変えます。
stroke関数以前に実行された描画については色を変えません。

\subsubsection{塗りつぶしの色を変える:fill(float)/fill(float,float,float)}
前項で示した図形のうち、色指定にfillを含むものの色を変えます。
例によって色指定方法は2種類あります。

\verb#fill(明度);#


または

\verb#fill(R,G,B);#

この関数は\bold{fill関数以降の図形関数の色だけ}を変えます。
fill関数以前に実行された描画については色を変えません。

\subsubsection{線を表示しない:noStroke()}
次にstroke関数が呼ばれる(実行される)まで、
色指定にstrokeを含む関数で線を描きません。

\verb#noStroke();#

例によって、noStrokeが呼ばれた後の図形関数にのみ影響します。

ちなみに、noStrokeが有効になっていると、line関数で何も描画されません。

\subsubsection{塗りつぶしをしない:noFill()}
次にfill関数が呼ばれる(実行される)まで、
色指定にfillを含む関数で塗りつぶしをしません。

\verb#noFill();#

例によって、noFillが呼ばれた後の図形関数にのみ影響します。

ちなみに、noFillが有効になっていると、text関数で何も描画されません。

\subsection{問題}
二歩目を歩けているかどうか、確認の問題を出します。
次の画面を出力してください。
\fig{2-2}{二歩目の問題}

関数は並べれば上から下に実行されます。
どのように書けば図形がうまく隠れるか、
どうすれば綺麗に塗り分けられるか、
いろいろ考えてみてください。

位置は完全に同じでなくてもいいですが、
テキストの表示、隠れ具合はだいたい同じにするのが良いです。

色の値は0と255しか使ってません。
適当に探してください。
\end{document}

